% !TEX encoding = UTF-8 Unicode
\documentclass[12pt,oneside]{memoir}
\usepackage{geometry}
%\geometry{letterpaper}
\geometry{a4paper}
\usepackage{graphicx}

%% For highlight ranges of text marked with inspector comments
\usepackage{soul}

\usepackage{xcolor}
\usepackage[
    colorlinks=true,
    urlcolor=blue,
    linkcolor=blue,
    citecolor=blue,
    filecolor=blue,
]{hyperref}
\usepackage{memhfixc}
\usepackage{natbib}
\setcitestyle{authoryear,open={(},close={)}}


\setcounter{page}{1}
\pagenumbering{roman}

\title{Introduction: Resurgence of the People}
\author{Sam Popowich}
\date{2020-09-20}

\usepackage{palatino}

\begin{document}

\maketitle
\clearpage

\newpage
\setcounter{page}{1}
\pagenumbering{arabic}

\mainmatter

\section*{Two Canoes}

Identity politics, along with authoritarian populism and far-right extremism, climate crisis, and the pandemic, is one of the major political issues of the day. Each of them challenges the orthodoxies of liberal political theory in a way we might think of as a resurgence of the people in political discourse. But ``the people'' cannot be considered a single homogeneous mass: debates over whether we are living in an ``anthropocene'' or a ``capitalocene'' \citep{Moore2016} hinges on whether anthropogenic climate change is due to the activity of some or all of humanity; populism  and right-wing extremism appeals to a tightly defined and exclusionary populus (nation, race, or Volk); the pandemic has shown how a phenomenon which ought to be universal in fact has differential effects based on race, and the social effects of quarantine and lockdown are also being felt by women more than men. The BlackLivesMatter protests which returned with a vengeance in the summer of 2020, and the intersectionality of the police/prison abolition movement, all indicate that any purported universalism of liberal theory is no longer tenable.

In 1995, in the middle of Canadian debates around multiculturalism, Quebecois nationalism, and Indigenous sovereignty, James Tully's book on modern constitutionalism, Strange Multiplicity (1995), drew inspiration from Haida artist Bill Reid's sculpture, The Spirit of Haida Gwaii. Tully saw in it a ``symbol of the age of cultural diversity'': Reid's sculpture depicts a crew of diverse beings drawn from Haida culture and the encounter with Europeans paddling a canoe, ``squabbling and vying for recognition'' \citep[24]{Tully1995}. Tully makes Reid's crowded canoe a metaphor for the pluralistic societies of what Tully saw in the mid-90s as a ``genuinely post-imperial age'' (17) after the collapse of the Soviet Union and the apparent triumph of liberal-democratic constitutionalism. The passengers in the canoe are gathered around a central figure, ``holding the speaker's staff in his hand{\ldots} the chief or exemplar, whose identity{\ldots} is uncertain'' (18). For Tully, the chief's position is one of mediation or reconciliation, just as liberal philosophy has been, since Locke, one of tolerance, pluralism, and the peaceful mediation of competing interests. However, in the twenty-five years since Tully's book was written, the pacific horizons of liberal constitutionalism have become increasingly fraught. The continuation of settler-colonial violence, not least in the Murdered and Missing Indigenous Women and Girls (MMIWG) crisis in Canada, as well as the resurgence of right-wing fundamentalism, populist governments, the reinstatement of intolerance and oppression based on race, gender, sexuality, class and disability - all in the context of social and economic crisis - all serve to make Tully's interpretation of Reid's sculpture problematic. It is hard for us now to take the chief seriously,  ``listen[ing] attentively to each [passenger], hoping to guide them to reach an agreement, without imposing a metalanguage or allowing any speaker to set the terms of the discussion'' (24) given the very real expressions of government contempt for ordinary citizens, state-capital relationships, racialized and gendered violence, and the differential effects of ongoing capitalist crisis, not excluding the global pandemic of 2020.

In the face of these crises of civil society, it is tempting to reject Tully's liberal conception of sovereignty and opt instead for a Hobbesian model of bellum omnia contra omnes limited only by the sovereign power of Leviathan. But this choice is a false one, as Hardt and Negri suggest at the beginning of Empire, published five years after Strange Multiplicity in the wake of the 1999 anti-globalization protests. There is a third constitutional option, one which is often dismissed out of hand because, as we will see, it relies upon an immanent, unruly creativity and self-directed activity that is a challenge both to the Lockean and Hobbesian constitutional orders. And just as there is an alternative to Tully's model of ``post-imperial'' constitutionalism, so there is an alternative to Reid's sculpture perhaps more appropriate to our contentious age.

	Kent Monkman is a two-spirit Cree artist, born in Ontario and raised in Winnipeg, Manitoba, a city to which we will have cause to return. Monkman's paintings have struck a chord with their provocative ``reconfigurations'' of classical European forms, motifs, and images in the service of ``challenging the visual narratives upon which Western expansion and `settlement' by Europeans is historically based'' \citep[181]{Elston2012}. Echoing Tully's understanding of ``hidden constitutionalism'' surviving within dominating, hegemonic constitutions, Monkman's work provides ``an example of the dynamic, ongoing practice of continued Indigeneity within supposedly conquered landscapes''. 
	
Monkman's 2019 painting, Resurgence of the People, depicts, like Reid's sculpture, a crowded canoe; only in this instance Monkman appears to reject Tully's conciliatory understanding of the politics of recognition in favour of a self-affirmation of the multitude in all its diversity. In Monkman's canoe, while there is still a central figure, the ``passengers'' (the term seems inappropriate here) are not gathered around pressing their case for recognition. Rather, the canoe is filled with people of colour of various ages and genders all looking after one another, not in a struggle for recognition, but in mutual support and care, indicating an alternative perspective from Reid's, and opening up a conception of politics broader than the democratci constitutionalism which is Tully's focus.

This alternative perspective, a rejection of both Hobbes' and Locke's conceptions of sovereignty, is made explicit in the presence of the central figure of Monkman's painting. Gone is the enigmatic chief who ``must act like a mother in caring for the common good if s/he is to secure respect and authority'' \citep[25]{Tully1995}. The need to secure respect and authority is a remnant of the false choice, a holdover of the need for constitutional sovereignty and centralization. In Resurgence of the People, Reid's chief has been replaced by Monkman's wild and exuberant alter-ego, Miss Chief Eagle Testickle, ``a time-traveling, shape-shifting, supernatural being who reverses the colonial gaze to challenge received notions of history and Indigenous peoples'' \cite{MonkmanBiography}.

While Reid's chief accords or grants his recognition to the passengers in the canoe, indicating the accommodation of diverse identities in a sovereign constitution, Miss Chief offers her own exuberant affirmation that overflows the limits of the canoe, suggesting an unruly and irrepressible power that does not depend either on recognition or on a constitution. Similarly, the others in the canoe are not looking to Miss Chief for recognition or leadership, but are living their lives, helping each other without concern for constitutional niceties, but all fully aware that they are living in the same boat. While in Reid's sculpture ``the boat goes on forever anchored in the same place'' \citep[33]{Tully1995}, betraying the desire of liberal politics for an orderly, predictable, and profitable future, in Monkman's painting the canoe is thrusting powerfully forward through choppy and uncertain seas. There is an unbridgeable gulf between the self-determining forward motion of Monkman's canoe, and the static, rockbound violence of settler-colonialism. The resurgence of the people is their unruly momentum, controlled by the determined paddling of Indigenous men and women, unconstrained by the power of a constitution, a momentum that turns away from the white, patriarchal, capitalist state to make its own way in the world. Miss Chief does not need anyone's recognition; rather, it is the white men stuck on the rock with their weapons who clamour for recognition: of their authority, their power, their capacity for violence.
The new politics of identity differs from the older politics of recognition precisely in this insistence of self-affirmation that transcends a uniform, egalitarian conception of rights and a procedural, discursive conception of democratic process. It takes seriously the incommensurable, the irreconcilable, and the non-dialectical tensions, antagonisms, and contradictions of contemporary social relations. This insistence challenges hegemonic ideas of the state, of democratic participation, of liberty, and of citizenship, all of which play out in current controversies and debates within Canadian librarianship.

[IMG: Resurgence of the People]

What is the connection of identity to constitutionalism in the Canadian context? The work of two philosophers - Charles Taylor and James Tully - is especially relevant here. Taylor has been a major figure in Canadian public intellectual life since the 1970s and Tully, while perhaps less well-known, has reframed Taylor's politics of recognition in constitutional terms in order to develop a constitutionalism proper to Canada's multicultural, multinational, and polyethnic society. Both Taylor and Tully come out of a tendency within liberal thought to give more weight to community bonds and collective identity (communitarianism). They were opposed to both the individualistic liberalism of Rawls, Dworkin and others, as well as the ``atomism'' \citep{Taylor1985} of libertarians like Robert Nozick. Neither Taylor nor Tully refers to more left-wing social and political theory (understandable in the context of the post-Soviet ``end of ideology'' and the triumph of liberal democracy), but this rejection limits how far they are able to understand and accommodate real challenges to liberal orthodoxy, such as those in existence today.
In order to properly contextualize Taylor and Tully's political theory, we have to understand the way liberalism responded to the critique of the new social movements, post-colonialism, and the worker-student revolts of the late 1960s. The liberal-communitarian debate, out of which Taylor's politics of identity developed, can be understood as a debate within a broader liberalism about the best way to respond to that critique, under pressure from the ``communitarian'' demands on the one side, and the neoliberal recuperation of those demands on the other.

\section*{1968: A New Kind of Revolution}

\begin{quote}
``We all want to change the world.'' The Beatles, ``Revolution'' (1968).
\end{quote}

If, as historian Eric Hobsbawm suggests, 1968 never looked like it could or would be the revolution, it nonetheless constituted a revolution. Hobsbawm argues that while student revolts in 1968 and 1969 were virulent, the students alone could never lead a mass movement against capitalism \citep[298-299]{Hobsbawm1994}. What Hobsbawm misses, however, is that not only were students allied with workers in the 1968 revolts \cite{Feenberg2001}, but that the worker-student revolt was part of a much broader social upheaval, a ``resurgence of the people'' once the period of capitalist reconstruction after the Second World War had been accomplished. Twenty years of social peace, capitalist development, and individual repression in the name of improved standards of living did not mean, as Hobsbawm claims, that ``revolution was the last thing in the minds of proletarian masses'' \citep[299]{Hobsbawm1994}. Rather, the revolutionary impulse adopted what today we would think of as an intersectional approach \citep[7]{Taylor2017}. The orthodox Marxist revolution against class exploitation led by Communist Parties and unions had become suspect or been rejected after 1956 \citep{Hall1990}, and in place of that revolution there arose a new revolutionary wave: polyvalent, pluralist, post-colonial, extra-parliamentary, and focused on civil rights. 

The context of this new uprising were the many geopolitical and social changes taking place in the post-war world beginning in the mid-1950s. The Algerian War of Independence (1954-1962) gave way to the Vietnam War (1955-1975). The Campaign for Nuclear Disarmament (CND) marches began in 1958 (and introduced Gerald Holton's ``peace sign'' to the world). The Selma-to-Montgomery marches of 1965 marked a turning point in the American Civil Rights movement. The Mouvement de libération des femmes was founded in 1968 by Antoinette Fouque, and in the same year feminists staged a protest agains the Miss America pageant: these two events are seen as part of the development of second-wave feminism. In 1969, the ``Stonewall Riots'' marked a turning point in what was then called the gay liberation movement, though as historian Susan Stryker points out, the riots were not the first of their kind. 

\begin{quote}
Gay, transgender, and gender-variant people had been engaging in violent protest and direct actions against social oppression for at least a decade by that time. Stonewall stands out as the biggest and most consequential example of a kind of event that was becoming increasingly common rather than as a unique occurrence. By 1969, as a result of many years of social upheaval and political agitation, large numbers of people who were socially marginalized because of their sexual orientation or gender identity, especially younger people who were part of the Baby Boom generation, were drawn to the idea of ``gay revolution'' and were primed for any event that would set such a movement off. The Stonewall Riots provided that very spark, and they inspired the formation of Gay Liberation Front cells in big cities, progressive towns, and college campuses all across the United States. \citep[82]{Stryker2017}
\end{quote}

Cultural theorist Stuart Hall argues that between 1964 and 1968, the world turned. The Civil Rights movement in the US gave strength to Black struggles in the UK in the context of a broader and deeper challenge to the post-war compromise between capital and labour that relied on assimilation and repression to keep the social peace. Hall sees this period as one in which the ``great consensus of the 1950s'' was challenged, and when the state and the ruling glasses began to understand that what appeared merely as anti-establishment childishness (we can think of the Beatles and Monty Python here) or a fad for ``permissiveness'' was in fact ``something worse than that - something close to an organized and active conspiracy against the social order'' \citep[149-150]{Hall1978}.

David Harvey, in his Brief History of Neoliberalism argues that it was the repression of individual and collective needs and desires in the name of post-war reconstruction and prosperity that led to the explosion of new social movements and new social and political demands. The neoliberal project - based on free market fundamentalism and individual consumer choice - recuperated the energies released in 1968 and 1969 to further its own political and economic project in the early 1970s \citep[10]{Harvey2005}. Indeed, the period between 1968 and the collapse of Bretton Woods in 1973 can be understood as the transition period to neoliberalism, paving the way for the election of ``neoliberal'' leaders like Thatcher in 1979 and Reagan in 1980. Nevertheless, despite neoliberal recuperation, the energy released, the new social, economic, and political demands singled a transition to a modern conception of collective identity, belonging, and obligation the effects of which we are still dealing with today.

The generalized crisis of 1968 set off two particular interlocking processes. The first was a re-evaluation of both liberal and Marxist political theory; the second was the beginning of an attempt by the Canadian government to deal with the issues raised by the struggles of the 1960s. This attempt was named ``the just society'' by Canadian politician Pierre Elliot Trudeau in 1968. We will deal with both of these processes in turn.

\section*{Political Theory and the Crisis of Representation}

If the invasion of Hungary and the Suez crisis in 1956 had fostered the creation of the ``first New Left'' as a ``third political space'' between Stalinism and Imperialism \citep[117]{Hall1990}, the rise of new social movements in the 1960s caused a further disillusionment with traditional Communist approaches to activism and organization. Stuart Hall writes that ``the New Left belonged to the same conjuncture as CND. It was the product of the same decay in the `relations of representation' between the people, the classes and the parties'' that became more pronounced in the 1970s and 1980s \citep[135]{Hall1990}. This decay also fostered the autonomist movement among Italian Marxists beginning in the 1950s and reaching its peak in the ``hot autumn'' of 1969. In his history of Italian automatism, Steve Wright notes that early autonomists ``all agreed that the growing moderation of the left parties and unions sprang first and foremost from their indifference to the changes wrought upon the Italian working class by postwar economic development'' \citep[20-21]{Wright2002}. Distrust of the ``traditional'' organizations of working-class representation - the Italian Communist Party (PCI) and the unions - led to the formation of an ``extra-parliamentary left'' in 1969 \citep[126]{Wright2002} and eventually to the Red Brigades, the murder of Aldo Moro in 1978, and the arrest of Antonio Negri in 1979.

The crisis of political representation caused two significant shifts in left political theory. On the one hand, it provoked a search for radically democratic non-representational forms of politics that could account for difference without resorting to the flattening effects either of post-war liberalism or Stalinism \footnote{Indeed ``difference'' became an important signifier at this time: Derrida's ``différance'' was introduced in 1963, and Deleuze's Différence et répétition appeared in 1968}. On the other hand, it challenged the primacy of the Hegelian/Marxist dialectic - with its logic of inexorable and teleological progress - and spurred a search for a political ontology that was dynamic but remained open rather than closed (or foreclosed) by a pre-determined historical goal. Both of these requirements were met in the rediscovery of Spinoza by philosophers and political theorists in the 1960s \footnote{Negri cites Gueroult's two-volume Spinoza (1968), Matheron's Individu et communauté chez Spinoza (1969), Deleuze's Spinoza et le problème de l'expression (1968) and Spinoza (1970), as well as  Macherey's later Hegel ou Spinoza (1977)}. In a recent evaluation of his encounter with Spinoza, Negri argues that 1968 marked a moment when both the hegemonic politics of repression and the Marxist politics of disciplined revolt gave way to a ```good moment'{\ldots} of affirming democratic thought and encouraging struggles open to the desire for happiness'' \citep[vii]{Negri2020} that resonated in the philosophy of Spinoza. Spinoza provided not only a non-representational, directly democratic conception of the multitude and constituent power (which became the foundation of Negri's own political thought), but also rejected what Negri and others felt to be the totalizing closure of the Hegelian dialectic: 

\begin{quote}
Materialism, seen through the epistemological and ontological lens of Spinozism, was able to abandon its traditional foundation dialectic and to embark on a project that was simultaneously constitutive and subjective. Thus Spinozism corresponds to a call for insurrection and to the new figure of class struggle that, from 1968 on, was no longer willing to squeeze through metaphysical straits towards teleological destinations. \citep[vii-viii]{Negri2020}
\end{quote} 

The effects of 1968, then, resonated throughout the world of radical left political theory, including Marxism, making space for new theoretical and practical strategies appropriate to the neoliberal conjuncture. But liberalism, too, was challenged by the new social movements and the threat to the social order. We can trace once such challenge in John Rawls' position on the Vietnam War and his attempt to deal with it in his Theory of Justice (1971).

John Rawls was vocal in his opposition to the Vietnam War, which he considered fundamentally unjust. He took part in an anti-war conference in 1967 and taught a course called ``The Problems of War'' in the spring of 1969. In his biography of Rawls, Thomas Pogge claims that there were two main social or political questions to be asked in relation to the war: ``what flaws in [Rawls'] society might account for its prosecuting a plainly unjust war with such ferocity, and what citizens might do to oppose this war'' \citep[19]{Pogge2007}. Rawls ascribed the flaws in American society to the unequal distribution of wealth and political power, and the solution to the problem in Rawls' view was to ensure that ``those similarly endowed and motivated should have roughly the same chance of attaining positions of political authority irrespective of their economic and social class'' \citep[225]{Rawls2005}. To the second question, Rawls considered it vital that opposition to the war - indeed, any civil disobedience or conscientious objection - be recognized as an expression of minority rights against the domination of the majority \citep[19]{Pogge2007}.

In the development of these ideas, which formed the background to A Theory of Justice, Rawls relies on both an individualist social ontology and a conception of social relations that pits individual against society. The individualist social ontology is part of the common stock of liberal political thought and goes back to the social contract theory of Rousseau, Hobbes, and others. Connected with this is the idea that individuals are the ontological foundation of society only coming together out of necessity - political in the case of Hobbes, economic in the case of the classical political economists - but who find their individuality constantly under threat by a diametrically opposed ``society'' confronted as something alien and external to them.
Despite Rawls' attempt to deal with some aspects of the social upheaval of the 1960s, his individualist ontology, his methodological reliance on the ``original position'' and the ``veil of ignorance'' (which saw identifying factors like race, class, or gender as something which could simply be bracketed off for the purposes of ethical decision-making) were contrary to the demands, positions, and strategies proposed by the new social movements, including anti-war activists. For Rawls, the original position required for his theory of justice entailed an individualism in which ``no one knows his place in society, his class position, or social status'' \citep[12]{Rawls2005}, which he explicitly connects to the ``state of nature'' of social contract theory. The individualism of Rawls' position and his view that communal bonds and obligations were irrelevant - if not anathema - to justice was elaborated by liberal thinkers like Ronald Dworkin, but provoked an alternative perspective within liberal thought known as ``communitarianism''. The liberal-communitarian debate tried to find the right balance between individualism (especially individual rights) and the sense of communal belonging and obligation that had been raised for political theory by the events of 1968. 

The relationship of individual to collectivity lies at the heart of this debate. Dworkin argues that liberals seek a balance between individual and social rights: ``They support racial equality, and approve government intervention to secure it, through constraints on both public and private discrimination in education, housing and employment. But they oppose other forms of collective regulation of individual decision{\ldots} Democracy is justified because it enforces the right of each person to respect and concern as an individual; but in practice the decisions of a democratic majority may often violate that right, according to the liberal theory of what is right'' \citep[122/134]{Dworkin1978}. Communitarians and libertarians come down on either side of this position, either insisting on a stronger normative weight for community, or on the absolute priority of individualism. 

	Taylor himself took on the libertarian side in an article simply called ``Atomism'' \citep{Taylor1985}. Atomism, for Taylor, describes the individualism that connects the social contract theory of the seventeenth century with contemporary doctrines that ``inherited a vision of society as in some sense constituted by individuals for the fulfilment of ends which were primarily individual{\ldots} or which try to defend in some sense the priority of the individual and his rights over society, or which present a purely instrumental view of society'' \citep[187]{Taylor1985}. The core of Taylor's critique of atomism is the idea that individual rights do not entail either belonging to or an obligation to the society or community in which we live. A feeling of belonging or obligation may be felt by an individual, but such feelings do not carry the normative or juridical weight of individual rights. For Taylor, however, community and society fosters individual autonomy, at least in part. If liberal individualists believe that individual autonomy (i.e the ability to choose a self-determined course of action) is worth instilling and preserving, then some sense of belonging and obligation is presented: liberal individualists must defend the society that best fosters individual self-determination, if nothing else, because they would wish their children and subsequent generations to also be self-sufficient individuals.
	
We can see here the genesis of Taylor's critique both of liberal individualism and libertarian atomism: the problem with a sole focus on individual rights is that it rejects any sense of social bond understood either as belonging or obligation. Indeed, the liberal-communitarian debate can be framed in terms of whether the framework of individual rights is sufficient to address collective demands -- such as multiculturalism, Indigenous sovereignty, or Quebecois nationalism -- in the form they have taken since the late 1960s. Put this way, communitarians like Taylor answer in the negative, ``liberals'' like Dworkin and Will Kymlicka answer in the affirmative, and libertarians like Robert Nozick reject the idea of collective demands and obligations outright.

	However, Taylor's communitarianism, while taking collective rights, belonging, and obligations seriously, still adopts the ``methodological individualism'' common to all three forms of liberal thinking. He rejects the idea of an ``autarchic'' individual living in a state of nature, but he still sees society as composed of autonomous individuals engaging in free choice, only externally related to other individuals or to society itself. Indeed, liberal society, in Taylor's view, is the only society that can produce individuals capable of understanding alternatives and making free choices. In order for freedom to mean anything, individual autonomy must be fallible (though the failure to achieve or exercise autonomy would not, in Taylor's view, be sufficient to deny someone their rights). It is only the traditional set of liberal institutions which serve to instill, support, and maintain individual autonomy. ``The free individual or autonomous moral agent,'' Taylor writes, ``can only achieve and maintain his identity in a certain type of culture'', and the facets of this culture are ``carried on in institutions and associations which require stability and continuity and frequently also support from society as a whole'' \citep[205]{Taylor1985}, by which Taylor means liberal cultural and political institutions which allow free, autonomous individuals to learn about different choices and positions and choose their commitments freely. But Taylor reiterates the typical liberal dualism between individual and society: ``I am arguing that the free individual of the West is only what he is by virtue of the whole society and civilization which brought him up and which nourishes him'' \citep[206]{Taylor1985}. 
	
	Taylor's recognition of the necessity of at least some level of social support in the maintenance of individual freedom and autonomy underpins his rejection of Nozick's libertarianism, because the very freedom of the individual ``cannot be concerned purely with his individual choices and the associations formed from such choices to the neglect of the matrix in which such choices can be open or closed, rich or meagre'' \citep[207]{Taylor1985}. Taylor's dualism leads him to the defense of bourgeois democratic political institutions: ``The identity of the autonomous, self-determining individual requires a social matrix, one for instance which through a series of practices recognizes the right to autonomous decision and which calls for the individual having a voice in deliberation about public action'' \citep[209]{Taylor1985}. In this sense, Taylor's communitarianism becomes a mouthpiece for the liberal state. Taylor's response is not only directed at Nozick, but at Dworkin's liberal proceduralism -- the idea that the state best guarantees the individual pursuit of the good life by remaining neutral or agnostic about any particular conception of the good, and only operating on society through ``neutral'' algorithms or procedures. Proceduralism was one target of the liberal-communitarian debate, for example in Michael Sandel's critique of ``the procedural republic'', which he considered was leading to a society devoid of social bonds, obligations, or belonging (Sandel 1984). Taylor's intervention tried to bring a social theory to bear on both Dworkin's proceduralism and Nozick's libertarianism which would allow for social responsibility while still maintaining a commitment to liberal political institutions and the capitalist state itself.
	
Taylor's communitarian social theory is still based on a separation between individual and society, in which an individual is formed in the ``matrix'' of social structures and institutions until they are able to act autonomously with freedom of choice. The idea of freedom of choice, however, mystifies the ideological reproduction of capitalist society and obscures -- as Dworkin's proceduralism does -- the very normative commitments capitalist society (and liberal theory as its mouthpiece) subscribes to: individual rights, private property, the market, white supremacy, and social class to name a few. The structures and dynamics of capitalist society -- including its mode of production and conception of private land ownership, which automatically precludes Indigenous socio-economic formations -- only allow the kind of political deliberation Taylor thinks is vital up to a certain point. In his work on the politics of recognition, Taylor limits his concerns to cultural recognition, that is the recognition of Indigenous, Quebecois, or immigrant cultural expressions, so long as they do not challenge the foundations of capitalist society (e.g. private property, etc.). Coulthard has critiqued Taylor's politics of recognition by arguing that recognition in Taylor's theory is based on a presumed equality between recognizing subjects, an equality that can never exist in settler-state/colonial relationships \citep{Coulthard2014}. This inequality means that recognition is, in fact, requested by the less powerful and either granted or withheld by the more powerful, according to the settler state's need for self-protection (it is for this reason that Coulthard turns to Fanon's rejection of Hegel's initial theory of recognition in a colonial context).

	Besides the issue of inequality, Taylor's politics of recognition stands or falls on his dualistic conception of individual identity and society. Drawing on his theory of identity and individualism developed in ``Atomism'' and Sources of the Self \citep{Taylor1989}, Taylor argues that it is the external confrontation between a self and society that produces a fully-functioning, autonomous, self-directed individual:

\begin{quote}

On the social plane, the understanding that identities are formed in open dialogue, unshaped by a predefined social script, has made the politics of equal recognition more central and stressful. It has, in fact, considerably raised the stakes. Equal recognition is not just the appropriate mode for a healthy democratic society. Its refusal can inflict damage on those who are denied it{\ldots} The projection of an inferior or demeaning image on another can actually distort and oppress, to the extent that the image is internalized. \cite[36]{Taylor1994}
\end{quote}

The expression ``unshaped by a predefined social script'' should give us pause. Taylor presumes that a society composed of free, autonomous individuals is equally free with respect to the social structures, dynamics, and relations it imposes on its members. We are reminded here of Marx and Engels' critique of the idealism of the Young Hegelians, who believed that simply by exchanging one set of phrases for another they were effecting real change in the world \citep[34-36]{MarxEngels1976}. If the world truly lacked a ``predefined social script'', then such freedom could perhaps be possible, but historical materialism is founded on the idea that ``the tradition of all the dead generations weighs like a nightmare on the brain of the living''. Indeed, it is hard to see how liberal institutions could play the socially-reproductive role Taylor assigns to them if they did not, in some sense, predetermine a social script (which of course they do)\footnote{In The Accumulation of Capital, Rosa Luxemburg makes a similar point about the need to recognize that a given cycle of capitalist reproduction only makes sense within a chain of prior cycles}.
A particular outcome of the liberal-communitarian debate was a focus on multiculturalism, rights, and citizenship within multinational and polyethnic states like Canada. Charles Taylor attempted to construct a communitarian social ontology and theory of identity-formation in opposition to the traditional liberal individualism, and made that ontology the basis of his intersubjective ``politics of recognition''; James Tully developed the politics of recognition into a constitutional theory appropriate for the post-1968 conjuncture; while Will Kymlicka defended the primacy of individual rights even in the context of ``multicultural citizenship''. These theoretical positions were not formed in an abstract vacuum, however, but responded to particular moments in Canadian political history between 1968 and the mid-1990s. Indeed, as we will see, we can understand Taylor's, Tully's, and Kymlicka's political theory as made possible by the Canadian state's inability to deal adequately with the social demands and political energies released in the aftermath of 1968.

\section*{Canadian Politics, Identity, and Belonging}

There are two perennial problems in Canadian politics: Indigenous sovereignty and Quebec nationalism. These two problems are difficult for liberalism - hegemonic within Canadian politics - to deal with, because they raise questions not only of cultural identity and belonging incommensurate with liberalism's universal individualism, but also challenge the isolation, atomism, and alienation of capitalist society. Indeed, Indigenous conceptions of land stewardship and relations with the earth directly challenge the extractive requirements of the capitalist mode of production itself.

Indigenous activism received fresh impetus in the late 1960s in the context of social and political demands on the part of a number of new social movements. The worker-student revolts of 1968 are perhaps the best-known example of resistance against the post-war consensus, but the anti-colonial wars in Algeria, Vietnam, and elsewhere, the civil rights movement, the advent of second wave feminism, and the explosion of gay and trans rights marked by Stonewall in 1969, formed a broad current of resistance to the ``embedded liberalism'' of post-war capitalist society. The National Indian Brotherhood, formed in 1968, represented a new sense of pan-Indigenous awareness in Canada. Prior to this period of social upheaval, however, liberal governments like that of Pierre Trudeau , sought to increase the universalism of post-war social policy. In his 1968 book, Federalism and the French Canadians, Trudeau merely expressed a liberal orthodoxy when he wrote that ``the state{\ldots} must seek the general welfare of all its citizens regardless of sex, colour, race, religious beliefs, or ethnic origin'' (Trudeau 1968, 4). And in formulating the influential ``Just Society'' policy in Canada, Trudeau ``rejected the notion that any group could be accorded a position separate from the rest of the population and was convinced that removing the legislated difference between Indigenous and other Canadians [i.e. the Indian Act] could cure Canada's `Indian Problem''' (Nickel 2019, 49-50). 

Based on this orthodoxy, the 1969 ``White Paper'' produced by then Minister of Indian Affairs, Jean Chrétien, proposed scrapping the Indian Act, eliminating ``Indian Status'' and fully assimilating Indigenous peoples into settler culture and society. However, Trudeau and Chrétien misunderstood ``Indigenous realities and how liberal concepts of individualism, freedom, and equality ran counter to Indigenous peoples' history, collective rights, and self-identification'' (Nickel 2019, 50). The phrase ``cultural genocide'', which proved controversial when it was used to describe the residential school system by the Truth and Reconciliation Commission in 2015 (TRC 2015, 1), was first used to refer to the effects of the White Paper policy by Harold Cardinal in his 1969 response to the federal position, The Unjust Society. Cardinal wrote that

\begin{quote}
The new Indian policy promulgated by Prime Minister Pierre Elliott Trudeau's government, under the auspices of the Honourable Jean Chrétien, minister of Indian Affairs and Northern Development, and Deputy Minister John A. MacDonald, and presented in June of 1969 is a thinly disguised programme of extermination through assimilation. For the Indian to survive, says the government in effect, he must become a good little brown white man. The Americans to the south of us used to have a saying: ``the only good Indian is a dead Indian.'' The MacDonald-Chrétien doctrine would amend this but slightly to ``The only good Indian is a non-Indian.'' (Cardinal 1969, 1).
\end{quote}

In Sarah Nickel's view, reaction to the White Paper in combination with Indigenous political action (including the formation of new associations and organization) ``provided an opening for Indigenous political discourse'' (Nickel 1969, 52) that was a new addition to the Canadian political landscape. This new pan-Indigenous political discourse led to an expansion of Indigenous resistance in the 1970s. According to Dene scholar Glen Sean Coulthard, such ``Indigenous anticolonial nationalism'' forced the Canadian government to abandon its policy of assimilationist cultural genocide and adopt ``a seemingly more conciliatory set of discourses and institutional practices that emphasize our recognition and accommodation'' (Coulthard 2014, 6). Legal struggle, such as the landmark Calder (1973) and Delgamuukw (1997) decisions on land and treaty rights, were important elements in this process. Indigenous resistance provides one strand in Canadian politics that leads to Taylor's formulation of a philosophically rigorous politics of recognition.

	The other strand is Quebecois sovereignty, which came to a head at the same time as the 1969 White Paper and its aftermath. The nationalist Front de libération du Quebec (FLQ), formed in the early 1960s, had carried out a number of attacks between 1963 and 1970, including the bombing of the Montreal Stock Exchange in 1969. In October 1970, the group kidnapped British Trade Commissioner James Cross and subsequently kidnapped and killed Quebec Labour Minister Pierre Laporte. Trudeau's response to the ``October Crisis'' was to trigger the War Measures Act and to institute martial law. The October Crisis marked a turning point in Quebecois sovereignty, as sovereigntists repudiated violence and focused on legal measures to advance the nationalist cause, such as the passing of the Charter of the French Language (known as ``Bill 101'') in 1977. Quebec constitutional sovereignty reached a high point with the referendum on independence in 1980. The referendum failed, but it underlined the importance of Quebec for Trudeau and his project to ``patriate'' the Canadian Constitution from Britain. The patriation of the constitution took place in 1982, amending the British North America Act and supplementing it with the Canadian Charter of Rights and Freedoms. This Charter followed the same liberal principles we have seen Trudeau adopt before. For Trudeau, constitutionally- protected language and education rights would undermine French-Canadian nationalism. One commentator has written that ``a bill of rights would fulfil one goal of the original Canadian Constitution -- to allow all its citizens to `consider the whole of Canada their country and field of endeavour''' (Weinrib 1998, ??).

The passing of the Constitution Act in 1982 ran into problems both with Quebecois and Indigenous sovereignty, and subsequent attempts to amend the constitution (the Meech Lake [1982] and Charlottetown [1992] accords) have been unable to deal adequately with these issues. The passing of the Canadian Multiculturalism Act in 1988 added another aspect to the problem of a universal Canadian identity, that of officially enshrined multiculturalism. The question of multiculturalism, multinationalism, and polyethnicity became part of the liberal-communitarian debate in Canada, taken up by Taylor, Will Kymlicka, James Tully and others. By the early 1990s, Taylor's intersubjective conception of identity and his communitarian critique of individualism coalesced in a ``politics of recognition'' that tried to explicate Canadian policy up to that point, and to lay the groundwork for a liberal reconciliation of national, ethnic, and economic contradictions for the future.

Trudeau's conception of the ``just society'', proposed at the same time as the assimilationist White Paper policy, underlines the particular contradiction within Canadian politics. A just society, for Trudeau (as for Rawls) is

\begin{quote}
one in which the rights of minorities will be safe from the whims of intolerant majorities. The Just Society will be one in which those regions and groups which have not fully shared in the country's affluence will be given a better opportunity. The Just Society will be one where such urban problems as housing and pollution will be attacked through the application of new knowledge and new techniques. The Just Society will be one in which our Indian and Inuit populations will be encouraged to assume the full rights of citizenship through policies which will give them both greater responsibility for their own future and more meaningful equality of opportunity. The Just Society will be a united Canada, united because all of its citizens will be actively involved in the development of a country where equality of opportunity is ensured and individuals are permitted to fulfill themselves in the fashion they judge best.
\end{quote}

Here we can see Trudeau's commitment an individual ontology and Dworkin's ``proceduralism'', a liberal universalizing equality which flattens out or assimilates difference, and an antagonistic understanding of the relations between individual (or minority) and the pressures of the majority. The contradiction lies in the commitment to support minority rights while extending a universal, proceduralist equality irrespective of individual or communal difference. The debates among Taylor, Tully, and Kymlicka were attempts to resolve this contradiction on the terrain of political philosophy, but such contradictions are irresolvable simply through intellectual effort or the discovery of new knowledge\footnote{This point is clearly made by Georg Lukács in History and Class Consciousness}; the contradiction can only be resolved through the transformations of the social and political relationships that give rise to it in the first place. 
However, the resolution of this contradiction is perhaps of less interest than the reasons for it. Perhaps the central claim of this thesis is that if we understand liberalism (in any of its forms) as a justifying and legitimating discourse of capitalism itself, then despite its claims to 1) universality and procedural equality and 2) an agnosticism towards any substantive conception of the good, liberalism does hold certain substantial social, political, and economic phenomena to be sacrosanct: private property, the capitalist class system, individualism itself, etc. Rawls' original position and Dworkin's proceduralism all make a virtue out of liberalism's individualism while supporting the ideological fiction that liberal society remains aloof from substantive notions of the good. Taylor's problem is that he takes them at their word, and merely attempts to give communal relations more weight within the liberal schema. But none of these thinkers recognize that if liberal society does in fact support and maintain some conception of substantive good, then - contrary to Dworkin's argument - certain ways of life must be excluded from the liberal regime (any way of life that challenges liberalism's substantive goods) and - contrary to Taylor's argument - cultural recognition can only be accommodated so long as it does not challenge capitalism itself. To put it another way, the recognition of minority or communal claims and demands is limited to cultural expressions and excludes political or economic ones.
Thus exclusion and assimilation are inscribed at the heart of liberal theory including communitarianism and Tully's democratic constitutionalism. In the empirical part of this thesis, I will look at the library as both mechanism of ideological reproduction, by which the substantive goods of liberal society (individualism, consumerism, etc) are promoted and instilled, but which also provides for an ideological cover for proceduralism and atomism by its appearance of being a communal, ``socialist'' institution. Besides operating as an organ of ideological reproduction, the library also upholds an exclusionary function laid on it as an institution of the liberal state. In order to sanction any challenge to the substantive goods of liberal society, the library must exclude all those who directly or indirectly challenge the goods of liberal society. 

What I want to focus on in librarianship are the related concepts of Intellectual Freedom (IF) and Social Responsibility (SR).  IF first arose in the context of the Second World War as a way for librarians to support what we now understand to be a proceduralist equality: if the liberal state is aloof from any particular conception of the good, then library collections must reflect that. The American Library Association's positions on IF over the years have staunchly defended the proceduralist view (almost, but not quite, approaching libertarian atomism). In the upheavals of the late 1960s, IF was formalized within the ALA with the creation of the Office of Intellectual Freedom in 1967 and the Freedom to Read Foundation in 1969. The new social demands we have looked at were also reflected in the ALA at the end of the 60s, with the development of the Social Responsibility Round Table, also in 1969. For American librarianship specifically, we can provisionally map IF to liberal positions and SR to communitarian ones, all conceived under the aegis of the First Amendment and freedom of speech.
In Canada, however, which does not subscribe to freedom of speech and the First Amendment, IF and SR must be understood a little differently. The politics of recognition became Canadian government policy long before it was formally defined by Taylor: according to Coulthard it became the unofficial policy towards Indigenous sovereignty after the failure of the assimilationist 1969 White Paper. Communitarian principles - and in particular the politics of recognition - have thus always been in conflict with Trudeau's ``just society''. The patriation of the constitution and the creation of the Charter of Rights and Freedoms in 1982 enshrined the liberal view, and made gestures towards cultural recognition, but because it could not accommodate non-cultural demands, the Canadian constitution has always been unstable and inadequate to Canadian realities. The regular resurgence of both Indigenous demands and Quebec nationalism - both of which conflict with both liberal universalism and with the requirements of the capitalist mode of production - attest to that.

For this reason, IF and SR in Canada are less distinct than they are in the US. IF already takes the politics of recognition seriously, merging some of SR's communitarian positions into itself. However, this brings Canadian IF positions into conflict with Canadian society's need to exclude, as IF becomes extremely ambivalent in that situation, subject to political discourse and debate differently than in the US. A good example of Canadian IF's ambivalence is the fact that in the transphobic speaker debates in 2019, Toronto City Librarian stopped speaking about IF and began speaking about ``free speech''. Free speech is not a concept in the Canadian juridical regime, and IF is the usual term within librarianship; Vickery Bowles' adoption of the rhetoric of free speech was an attempt to gain some theoretical solid ground in order to circumvent the ambiguity of IF in Canada.

In this way we can understand how exclusion is required by capitalist, settler-colonial Canadian society, and the role libraries play in maintaining that exclusion. The wrong kind of people, people who challenge the presumed equality and noble values of Canadian liberalism, or who challenge the social and economic requirements of capitalism, must be excluded from the library for ideological reasons. In the empirical chapters, we will look at the ways the exclusion of transgender people, poor people (mostly Indigenous), and surplus workers have been excluded from Canadian libraries in recent years. Additionally, I will propose an alternative form of democracy that would enable libraries to live up to their stated goals and intentions: Antonio Negri's anarcho-communist conception of constituent power.

\section*{Marx's Social Contract Critique}

What I propose is a contrapuntal reading of Canadian politics and political theory with Negri's Marxist-Spinozan alternative. Such a reading does not require a resolution, but points out and investigates points of convergence and divergence, relating aspects of each to elements of the other two. In order to make the application of Negri's political theory to Canadian politics of identity and community on the one hand, and Canadian liberal debates on the other, we need to touch briefly on Marx's critique of the individualism of the social contract theorists and classical political economists that provided the foundation for classical liberalism and the individualism of Rawls, Dworkin, and Kymlicka. 

In ``On the Jewish Question'', Marx argues that the development of the capitalist state sought to institute liberal ``universality'' by pushing questions of identity into the private sphere of civil society. The private sphere became, on the one hand, the sphere of individual beliefs and opinions, values, and culture, but on the other hand became the sphere of Hobbes' war of all against all. Only in the supposedly universal political sphere -- the sphere of the state and public participation -- were any kind of social bonds recognized, but those bonds were shorn of any specific quality, and recognized only the kind of universal connections that would later be enshrined in Rawls' original position and Dworkin's proceduralism.

	Political emancipation in the liberal and libertarian sense is, then, the ``decomposition'' (Marx's term) of people into individual identity on the one hand and abstract, universal citizenship on the other. ``The decomposition of man,'' Marx writes, ``into Jew and citizen, Protestant and citizen, religious man and citizen, is not a deception practiced against the political system nor yet an evasion of political emancipation. It is political emancipation itself{\ldots}'' (Marx 1978, 35-36). The same can be said of the ways that the lived experience of transgender people, the questions of Indigenous identity, Quebecois identity, or the identity of people of colour are forced out of the political sphere into the sphere of private interest and individual cultural expression. Questions which cannot be forced into the private sphere -- Indigenous land rights and non-capitalist modes of production, for example -- fall victim to the material and political priorities of the capitalist state.
	
	Taylor's politics of recognition tries to bridge the gap between the abstract, universalist concept of citizenship on the one hand and the collective belonging and obligation of civil society on the other. Civil society, in his view, and contra Nozick, is a sphere of reciprocity and relationality which needs to be accounted for in political decision-making. Ignoring these collective bonds is precisely what has led to the unstable condition of Canadian constitutionalism, unable to satisfy Quebecois and Indigenous demands, but unable simply to reject them.
	Indeed, Trudeau's insistence on a Canadian Charter of Rights falls into the same trap Marx analyzed in ``On the Jewish Question'':

\begin{quote}
Let us notice, first of all that the so-called rights of man, as distinct from the rights of the citizen, are simply the rights of a member of civil society, that is, of egoistic man, of man separated from other men and from the community{\ldots} [The question of civil rights] is a question of the liberty of man regarded as an isolate monad, withdrawn into himself. (Marx 1978, 42)
\end{quote}

Indeed, such civil rights, for Marx enshrine the individualistic isolation of human beings in liberal political theory: ``liberty as a right of man is not founded upon the relations between man and man, but rather upon the separation of man from man. It is the right of such separation. The right of the circumscribed individual, withdrawn into himself'' (Marx 1978, 42). The problem with Taylor's theory is that it sticks to this conception of identity as an individual development within civil society; that the individual comes to political society with requests or demands for recognition. Taylor's theory cannot in the end bridge the gap between an individualistic, isolated private sphere, and the abstract, universal (but social) sphere of public or political life. Taylor's individual is still caught in the war of all against all, only instead of looking to protect themselves, they demand recognition. Recognition takes the place of emancipation in Taylor's theory.

In his critique of Proudhon, Marx takes exception to the idea that in an originary state of nature ``everyone produces in isolation'' and at the same time finds already in existence a division of labour, exchange and exchange value. In Proudhon's view, economic life consists of such individuals making the free choice to come together in relations of exchange, but Marx argues that Proudhon gets things the wrong way around:

Economic categories are only the theoretical expressions of the abstractions of the social relations of production. M. Proudhon, holding things upside down like a true philosopher, sees in actual relations nothing but the incarnation of these principles, of these categories. (Marx 1955, 95)

The categories in which Proudhon understands political economy -- division of labour, exchange, and exchange value -- are produced by the very political economy he seeks to explain. Marx's famous expression of the relationship between technology and mode of production (``the hand-mill gives you society with the feudal lord'') is followed by a clear expression of historical materialism: ``The same men who establish their social relations in conformity with their material productivity, produce also their principles, ideas and categories, in conformity with their social relations'' (Marx 1955, 95). Individualism, too, rather than an unavoidable ontological fact, is an idea and category that depends on the social relations of the mode of production. As Marx argues, the bourgeois economists and political theorists take as a starting point what has really been the end-result of a particular history: the corrosive individualism of capitalist development is raised to an objective category in liberal political thought from the seventeenth century contractarians to the present day.

	In the Grundrisse, Marx makes this process clear, charging Rousseau's social contract with ``bring[ing] together naturally independent, autonomous subjects into relation and connection by contract'', as if ``the individual and isolated hunter and fisherman'' were a fact of nature (Marx 1973, 83). Nozick's atomism, Rawls and Dworkin's individualism, and Taylor's dualist communitarianism share with the contractarians a mistake in the order of things. Rather than understanding the individual as a category produced by the long centuries of capitalist alienation, they place the individual first both logically and temporally. Marx writes that ``in this society of free competition, the individual appears detached from the natural bonds etc. which in earlier historical periods make him the accessory of a definite and limited human conglomerate'' (Marx 1973, 83). Taylor, just as much as Smith and Ricardo, sees the isolated individual produced by capitalism ``as an ideal{\ldots} as the natural individual appropriate to [his] notion of human nature, not arising historically, but posited by nature''. The fact that Taylor sees human autonomy as fallible should not obscure the fact that he still sees isolated individualism as distinct from society, as coming to society in the way Proudhon described. Marx observed that this dualistic notion only came into being with the development of capitalist social relations: 

Only in the eighteenth century, in `civil society', do the various forms of social connectedness confront the individual as a mere means towards his private purposes, as external necessity. But the epoch which produces this standpoint, that of the isolate individual, is also precisely that of the hitherto most developed social{\ldots} relations{\ldots} Production by an isolated individual outside society{\ldots} is as much an absurdity as is the development of language without individuals living together and talking to each other. (Marx 1973, 84)

Social relations pre-exist the individual; there is never a moment of autonomy outside them. The independent, autonomous individual -- whether intersubjectively influenced, isolated, or autarkic -- does not enter into social relations from outside, as it were, but is constructed or produced by them. We can distinguish, then, between the various liberalisms which hold firm to a bedrock of individualism and individual rights (the individual's ``private purposes''), and those schools of social construction of which Marxism is one. Liberalism takes for granted that some form of individualism is the only true or correct social ontology because it takes the self-determined freedom to choose of the bourgeois individual as its starting point, just as Marx argued. 

Even within Marxism, there are debates over exactly how the construction of subjectivity or identity takes place. Althusser's structuralist conception of ideology and capitalist reproduction stands as an example. For Althusser, an ideological state apparatus ``hails or interpellates concrete individuals as concrete subjects'' (Althusser 2014, 190). Althusser distinguishes between an individual and a subject, arguing that

\begin{quote}
The individual is interpellated as a (free) subject in order that he shall submit freely to the commandments of the Subject, i.e. in order that he shall (freely) accept his subjection, i.e. in order that he shall makes the gestures and actions of his subjection `all by himself'. There are no subjects except by and for their subjection. (Althusser 2014, 269)
\end{quote}

Remnants of the liberal confrontation between individual and society remain in Althusser, but he argues that ideology recruits and transforms all individuals, and so the result is the same: no individual is left outside the structuring bounds of ideological and social relations. We can understand ``ideology'' here as a shorthand for the social construction of identity.

	Roy Bhaskar distinguished between four models of the relationship between individual and society, which calls the Weberian, Durkheimian, the dialectical, and the transformational. The Weber and Durkheim models are both liberal in the sense that the individual is confronted by society as an external object. In Bhaskar's view, Weber and Durkheim differed only in that where Weber saw individuals as determining the social order, Durkheim saw the social order as determining individuals. The dialectical approach was a model in which ``society forms the individuals who create society; society, in other words, produces the individuals who produce society, in a continuous dialectic'' (Bhaskar 2015, 32). Calling this model ``dialectical'' unfortunately implies that this is the social model adopted by Marx, but I think Marx's view was closer to what Bhaskar calls the ``transformational'' model. Bhaskar describes this model as follows:

People do not create society. For it always pre-exists them and is a necessary condition for their activity. Rather, society must be regarded as an ensemble of structures, practices and conventions which individuals reproduce or transform, but which would not exist unless they did so. Society does not exist independently of human activity (the error of reification). But it is not the product of it (the error of voluntarism). Now the processes whereby the stocks of skills, competences and habits appropriate to given social contexts, and necessary for the reproduction and/or transformation of society, are acquired and maintained could be generically referred to as socialization. It is important to stress that the reproduction and or transformation of society, though for the most part unconsciously achieved, is nevertheless still an achievement, a skilled accomplishment of active subjects, not a mechanical consequent of antecedent conditions. (Bhaskar 2015, 36)

Bhaskar is less interested in the consequences of this model for politics than for the philosophy of science, but the transformational social model has serious political implications. In my view, Antonio Negri, drawing on the work of Spinoza, adopts a form of this model in his productive theory of social relations. Spinoza's constitutive ontology sees being as a production, as the production of nature -- including human beings. But the creation of the human, social world -- of ``second nature'' is, as it is with Bhaskar, not a determinism, but an achievement of active subjects. Those active subjects, too, are produced (composed) by the world: ``Subjectivity is a composition, first physical and then historical'' (Negri 1991, 226). For Spinoza, collective unity -- the multitude -- is the basis of a political ontology, not individuals. Critiquing social contract and natural right theories, Negri writes that 

Spinoza's specific formulation [of natural-right] evades and rejects what seem to be the fundamental characteristics of natural-right philosophies: the absolute conception of the individual foundation and the absolute conception of the contractual passage. And opposed to these absolute fundamentals, Spinozian thought proposes a physics of society: in other words, a mechanics of individual pressures and a dynamics of associative relationships, which characteristically are never closed in the [Hegelian] absolute but, rather, proceed by ontological dislocations [i.e. revolutions]. (Negri 1991, 109)

Negri's communist politics -- his conception of constituent power -- derives from this productive notion of individual subjectivity. ``The passage from individuality to community does not come about either through a transfer of power or through a cession of rights; rather it comes about within a constitutive process of the imagination that knows no logical interruption'' (Negri 1991, 110). 

	It is in this sense, then, that the communal strength of the community -- constituent power -- is radical democracy. Individual people (what Negri calls ``singularities'') aren't really individuals since they are produced by the social relations of the multitude. Through their activity -- not ``free'' in an absolute, bourgeois, liberal sense -- they produce and reproduce society itself. Negri's Spinozan understanding of freedom is never absolute, it is freedom according to historical, material, and social necessity. For Negri, the absolute, unbounded democracy of the multitude is always faced with the necessary constraints of history and the artificial compulsion of the foreclosing and antidemocratic power of capital and the state. This explains why there are always certain normative goods that liberal theory -- no matter how communitarian -- must always protect and defend: they are the goods of racial, patriarchal capitalism. Individualism, private property, the class structure, racist and sexist structures of oppression, these are sacrosanct, since it is through them that the production of surplus value operates. 

What would a Negrean conception of subjectivity and politics mean for the constitutional questions that have plagued Canadian society since its inception? More broadly, what would it mean for the larger questions around identity and politics, such as trans rights, BlackLivesMatter, and crimes against Indigenous women and girls?

	In the first place, the Marxist conception of totality can be reframed as Spinozan monism: people in society must be conceived in terms of a radical equality. Such equality would not be numerical or quantitative but a pure axiomatic assertion. Equality in this form would require abandoning private property (for equality of access to resources), representative government (for equal voices in political participation), class, race, and gender as categories of inequality. The concept of a radical social equality would also strengthen the idea of social relations and bonds of belonging, responsibility, and reciprocity (so important in Indigenous thought and social understanding) and overcome the individualism -- libertarian, independent, or communitarian -- that underpins capitalist social, economic, and political theory. The challenge to representative institutions would also force a re-evaluation of liberal political institutions, open space for alternative political arrangements, and the challenge (from Indigenous land and economic practices, for example) to the mode of production itself.
	
	The productive social ontology which understands individuals as not somehow self-created and coming to society as to some external phenomenon, would recognize that identity is the nexus of historical, social, and individual forces, coming together in a necessary but uncompelled individuality. In this sense, Spinoza's anti-liberal conception of freedom -- as acting by necessity but not by compulsion -- would replace the bourgeois conception, which sees freedom and necessity as antagonistic principles.
	
	Individual identities as the nexus of history and society allows us to approach the concept of the multitude from the bottom, so to speak, from individual subjectivity, rather than from the top. But the monist conception of society built up out of the incredible richness of human difference becomes, in this way, a singular entity, capable{\ldots}

\section*{Chapter Outline}

In Part One, I will focus on Taylor's and Tully's politics of recognition. Chapter 2 will look at Taylor's intersubjective view of identity formation and contrast it with the individualism and atomism on the one hand, and Negri's ``productive'' understanding of identity-formation derived from Spinoza. Chapter 3 will turn to Tully's democratic constitutionalism, situating it within the context of the political theory of constituent power, and proposing Negri's Spinoza reading of constituent power as an alternative. Chapter 4 will look at the dialectic of virtue and fortune, or freedom and necessity, from the perspectives of Taylor, Tully, and Negri, and explore the ramifications particular conceptions of ``freedom'' have for Intellectual Freedom in librarianship. In Part Two I turn to libraries, focusing on the ways in which they enable and justify exclusion of the ``wrong'' kinds of people. In Chapter 5 I explore the Toronto Public Library room rentals to transphobic speakers and how the exclusion of transgender people's lived experience was justified by reference to Intellectual Freedom. I will connect trans rights here to the Marxist and Negrean critique of rights more broadly. In Chapter 6 I look at the security measures put in place at Winnipeg Public Library, justified by reference to Social Responsibility, but excluding poor and Indigenous people, against whom a broader security culture in Winnipeg has been aimed. This chapter will look specifically at Indigenous exclusion from their own land, and the ways land dispossession is a requirement of settler-colonial capitalism. In Chapter 7, I will look at precarious labour in librarianship, the concept of librarians as an aristocracy of labour, and the ways in which proletarianization of library workers leads to a reserve army of the unemployed, all exacerbated by political, pandemic, and climate crisis. 









 


\backmatter

\bibliographystyle{plainnat} 
\bibliography{Bibliography}
\end{document}

\bibliographystyle{plainnat} 
\bibliography{Bibliography}
\end{document}
