% !TEX encoding = UTF-8 Unicode
\documentclass[12pt,oneside]{memoir}
\usepackage{geometry}
%\geometry{letterpaper}
\geometry{a4paper}
\usepackage{graphicx}

%% For highlight ranges of text marked with inspector comments
\usepackage{soul}

\usepackage{xcolor}
\usepackage[
    colorlinks=true,
    urlcolor=blue,
    linkcolor=blue,
    citecolor=blue,
    filecolor=blue,
]{hyperref}
\usepackage{memhfixc}
\usepackage{natbib}


\setcounter{page}{1}
\pagenumbering{roman}

\title{Intellectual Freedom, Constituent Power, and the Politics of Recognition in Canadian Libraries}
\author{Sam Popowich}
\date{2020-06-02}

\begin{document}

\maketitle
\clearpage


\tableofcontents

% \listoffigures
% \listoftables

\newpage
\setcounter{page}{1}
\pagenumbering{arabic}

\mainmatter

\chapter{Introduction: Resurgence of the People}
\label{scrivauto:5}

\section{The Politics of Intellectual Freedom}
\label{scrivauto:6}

Intellectual Freedom (IF) is considered to be one of the core values of librarianship. Within the field it is held to be the equivalent of free speech or freedom of expression in constitutional contexts, and is connected to civil and human rights either through the First Amendment (in the American contexts) or the UN Declaration on Human Rights (in Commonwealth or international contexts). There is a dominant or hegemonic discourse within the profession which, like much else within librarianship, tends to understand IF and human rights more broadly as politically neutral and non-ideological. One of the goals of this thesis is to tease out some of the ideological and political assumptions and the commitments they entail. As Roy Bhaskar suggests in \textit{The Possibility of Naturalism}, periods of crisis tend to expose previously opaque generative social structures to make them amenable to analysis, and thereby to change \cite[52]{bhaskar-1979}. It is another goal of this thesis to provide a framework for a changed understanding both of the politics of recognition and of intellectual freedom. The current context, marked by pandemic and worldwide social unrest, requires an engagement with political theory in order to properly understand and transform librarianship itself.

Intellectual Freedom was enshrined in librarianship on the eve of the Second World War, with the creation of the Library Bill of Rights in 1938, modelled on the American Constitution \cite[147-152]{samek-if1}. The creation of the UN Declaration of Human Rights in 1948 added international juridical weight to the idea of intellectual freedom, and formed the basis of non-American IF statements, such as those developed by the Canadian Library Association (1974) \cite{cla-if}, the Canadian Federation of Library Associations (2016) \cite{cfla-if}, and the International Federation of Library Associations (1999) \cite{ifla-if}. Despite the centrality of IF within the profession, the concept has been periodically challenged, generally from a position known simply as ``social responsibility'' \cite{samek-if1}, which sought to use library collections and services to further the cause of social justice. Social Responsibility is often considered to run counter to an ``absolutist'' conception of IF \cite[89]{gorman-2000} which sees any social (i.e. non-technical) consideration of collections and services as censorship. This attitude runs alongside the purported neutrality of libraries and library work \cite{lewis-2008} and a discourse that argues that such neutrality is necessary for libraries to uphold and maintain democratic values and practices \cite{popowich-2019}.

Criticism of the absolutist-IF position has been part of a long history of a broader challenge to particular facets of library work, going back to criticisms of library subject headings \cite{berman-1971} the position of women within the field \cite{garrison-1972}, and the social or political neutrality of collection development in the ``Berninghausen debate'' of the early 1970s. \cite[4-5]{samek-if1}. It is no accident that critiques of librarianship focusing on social justice arose in the period after 1968: the transition to neoliberalism challenged many of the previously-held orthodoxies of the field. Since then, ``radical'', ``critical'' or ``progressive'' librarianship has continued to critique the profession by exposing the power relations built into classification systems \cite{olson-2002}, or the gendered and racialized culture of librarianship [\cite{lew-yousefi-2017, schlesselman-tarango-2017, chou-pho-2018}]. The current period of crisis as we transition out of neoliberalism into whatever comes next (\cite{bonfert-2020}) has brought a new round of challenge to the orthodoxies of the profession, with intellectual freedom being particularly contested, especially since the election of Donald Trump and the resurgence of the far right since 2016. 

A series of controversies around intellectual freedom have exposed deep rifts within the profession over what has long been considered one of its core values and commitments. The connection of intellectual freedom with human rights - for example by explicit reference to Article 19 of the UN Declaration of Human Rights - allows criticism of IF-absolutism to be dismissed as anti-human-rights. This simplistic dismissal has the effect of shoring up IF-absolutism and the - often unconscious or at least unacknowledged - political commitments entails by that perspective. In the Canadian context, the rise of the right-wing can be dated to around 2016, when University of Toronto professor Jordan Peterson refused to call students by their chosen pronouns following the passing of Bill C-16 which included gender expression and sexuality as protected characteristics in both the Canadian Human Rights Act and the Canadian Criminal Code \cite{peterson-2016}. The fallout from Peterson's notoriety, especially with the well-publicized case of Wilfrid Laurier TA Lindsay Shepherd the following year \cite{hutchins-2017}, was part of the larger controversies around free-speech on university campuses in Canada and the US \cite{mackinnon-2018}, for example around the deplatforming of right-wing provocateur Milo Yiannopoulos in 2018 \cite{Beauchamp-2018}. In addition to questions of free-speech, 2016-2018 saw the creation of far-right/white supremacist/populist movements and political parties, including the Sons of Odin (founded in Finland in 2015, first active in Canada the following year), the Proud Boys (founded 2016), the Canadian Nationalist Party (founded 2017) and the People's Party of Canada (founded 2018). 

Thes is the context in which the most recent round of challenges to IF-absolutism arose in the summer of 2017 beginning with Toronto Public Library (TPL) renting space to a group including  Paul Fromm, international director of the white-supremacist Council of Conservative Citizens \cite{quan-2015}, and Marc Lemire, former president of the Neo-Nazi Heritage Front \cite{craggs-2019}. The space was rented for a memorial service for Barbara Kulaszka, a Toronto lawyer who had defended many right-wing cases, including that of holocaust-denier Ernst Zundel.  Kulaszka has been ``credited with ensuring that no Nazi has ever been convicted for war crimes in Canada'' and it has been argued that ``because of her work, there are{\ldots} no Canadian laws against internet hate speech'' \cite{shakeri-2017}. The library defended the room-rental in the name of intellectual freedom and the protections of the Canadian Charter of Rights and Freedoms. City Librarian Vickery Bowles released a statement that: 
\begin{quote}
To deny access on the basis of the views or opinions that individuals or groups hold contravenes the Canadian Charter of Rights and Freedoms and the principles of intellectual freedom, both cornerstones of the library's mission and values. Sometimes in defending freedom of speech, it's very uncomfortable to be put in a situation where we are defending the rights of those whose viewpoints many consider to be offensive. However, it is at those times that we must be vigilant in protecting the rights of all. \cite{tpl-kulaszka}
\end{quote}

 Despite protests, the memorial went ahead, though under pressure from the mayor and city council the library did amend its room-rental policy to allow for the denial of a rental when the library 
\begin{quote}
reasonably believes the purpose of the booking is likely to promote, or would have the effect of promoting, discrimination, contempt or hatred of any group, p, hatred for any person on the basis of race, ethnic origin, place of origin, citizenship, colour, ancestry, language, creed (religion), age, sex, gender identity, gender expression, marital status, family status, sexual orientation, disability, political affiliation, membership in a union or staff association, receipt of public assistance, level of literacy or any other similar factor. \cite{tpl-revisions}
\end{quote}

This amendment to the room-rental policy will play a major role in the subsequent controversy over TPL's room-rental to a ``gender feminist critical'' group in 2019.\footnote{For her defence of renting space for the Kulaszka memorial, Bowles won the 2018 Les Fowlie Intellectual Freedom Award from the Ontario Library Association. The next year's winner was James Turk. We will have occasion to discuss Bowles and Turk in more detail later on.} 

The question of room bookings and hate groups remerged the following year when the American Library Association's (ALA)\footnote{While the ALA does not have official authority over the profession, as the oldest and largest library association, it holds considerable hegemony within the field, especially in North America. ALA is also the accrediting body for professional library education in Canada and the US.} Office of Intellectual Freedom attempted to explicitly include ``hate groups'' as protected by the Library Bill of Rights room-booking policy \cite{yorio-peet-2018}. The ALA membership pushed back against this inclusion and eventually overturned it, but it exposed the OIF as traditional guarantors of IF-absolutism within the profession. In 2019, the issue came up again, this time around Meghan Murphy, a controversial ``gender critical feminist'' who had been banned from Twitter in late 2018 for violating Twitter's policy on transphobia by misgendering another user \cite{brean-2019}. In January 2019, Murphy was invited by a local group to be part of a panel on ``Gender Identity and Women's Rights''. Murphy, like Peterson, objects to Bill C-16. Despite defending the library's decision to rent space for the panel in the same terms as Vickery Bowles did, Vancouver Public Library also changed their policy in the aftermath of protests. The amended VPL policy is intended to ``limit the likelihood that prohibited activities, including hate speech (as defined by law), will take place on Library premises, and to assist the Library in identifying events that may require additional preparation by Library staff'' \cite{vpl-policy}.
Despite the change to the policy, Vancouver Public Library was banned from marching in the 2019 Vancouver Pride Parade (the University of British Columbia was also banned for similar reasons) \cite{cbc-vpl, cbc-ubc}. Within the profession, the decision by the Vancouver Pride Society was decried as attempted censorship (by IF absolutists) or at best as a tension or conflict between values. In the VPL statement of values (\cite{vpl-values}), ``intellectual freedom'' is one value out of twelve. Two other values - ``community-led planning'' and ``community partnerships'' - indicate a collective view of the library's role in society. Indeed, many members of the LGBTQIA+ community in Vancouver, with whom the library had been building up community relationships over the past several years, felt betrayed by the decision to allow Meghan Murphy to speak. However, in a statement defending the room rental, VPL argued that rather than one value among many, to be balanced and integrated, intellectual freedom was rather the single overriding core value for libraries: ``the fundamental role of libraries as a place for free expression and intellectual freedom must be upheld''.
It is precisely in this ``tension between values'' approach that the absolutist nature of hegemonic IF becomes clear. Whenever values come into conflict, for any reason, Intellectual Freedom always wins out. Supporters of this view tend to fall back on legal and juridical frameworks, claiming that the First Amendment, the Canadian Charter of Rights and Freedoms, or the UN Declaration of Human Rights prevent libraries from following any other policy. There is, then, a clear hierarchy which places community values at the bottom - to be adhered to in times of stability and peace - and the constituted power of the liberal-democratic juridical framework at the top. One of the main goals of this thesis is to understand the relationships between that hierarchy and two distinct constitutional models. On the one hand, there is the recognition-based model which has been dominant in Canada since the 1970s and which was formalized within political theory in the early and mid 1990s. On the other hand, there is the theory of constitutional power elaborated by Antonio Negri. One of the main claims in this project is that consciously or unconsciously, Canadian librarianship's conception of IF is informed by a politics of recognition that tries to be progressive, but which ends up supporting and reproducing the structural inequities of settler-colonial, cisnormative, patriarchal capitalism. An alternative conception of IF, one informed by Negri's Marxist politics, might help provide a different way of thinking about IF which might help support and maintain projects of resistance and social justice within Canadian libraries.
In addition to the goal for librarianship, however, there is a broader goal in terms of political theory. We will look at the various critiques of the liberal-idealist politics of recognition formulated by people like Charles Taylor, Axel Honneth, and Will Kymlicka, but each of these critiques ends up seeming one-sided, connected to particular struggles (e.g. Indigenous sovereignty or trans rights). The application of Negri's general theory of constituent power to the critique of the politics of recognition will, I hope, prove to be a significant contribution to political theory. Before turning to the politics of recognition and its critique, however, I want to trace the connection of Intellectual Freedom in libraries to the classical liberal tradition, and to show how the politics of recognition and IF are connected via the UN Declaration of Human Rights.

\section{Liberal Philosophy and Information Ethics}
\label{scrivauto:7}

\backmatter

\bibliographystyle{plainnat} 
\bibliography{Bibliography}
\end{document}
