% !TEX encoding = UTF-8 Unicode
\documentclass[12pt,oneside]{memoir}
\usepackage{geometry}
%\geometry{letterpaper}
\geometry{a4paper}
\usepackage{graphicx}

%% For highlight ranges of text marked with inspector comments
\usepackage{soul}

\usepackage{xcolor}
\usepackage[
    colorlinks=true,
    urlcolor=blue,
    linkcolor=blue,
    citecolor=blue,
    filecolor=blue,
]{hyperref}
\usepackage{memhfixc}
\usepackage{natbib}
\setcitestyle{authoryear,open={(},close={)}}


\setcounter{page}{1}
\pagenumbering{roman}

\title{Constituent Power and the Politics of Recognition: The Case of Intellectual Freedom in Canadian Libraries}
\author{Sam Popowich}
\date{2020-07-29}

\usepackage{palatino}

\begin{document}

\maketitle
\clearpage


\tableofcontents

% \listoffigures
% \listoftables

\newpage
\setcounter{page}{1}
\pagenumbering{arabic}

\mainmatter

\chapter{Papers}
\label{scrivauto:5}

\section{Algorithms, Procedures, and Constituent Power}
\label{scrivauto:6}


\begin{quote}

In programming, we deal with two kinds of elements: procedures and data. (Later we will discover that they are really not so distinct.) 

- Abelson, Sussman, and Sussman, \emph{Structure and Interpretation of Computer Programs}

\end{quote}

\section*{Introduction}

Contemporary liberal thought shares with computer programming an artificial distinction between algorithm and data, form and content, a ``value-neutral'' instrumental view of social life, and one which takes seriously the normative, moral, and ethical content of living itself. The inability or unwillingness to overcome this false binary lies at the heart of technology company's inability to deal adequately with fake news, propaganda, or racism on their platforms: the algorithms are pure mathematics; their content is due to the ugliness of human life which has nothing to do with the algorithms. The dominant conceptions of liberal democratic institutions is eerily similar: due process and the rule of law function abstractly, and work properly as long as they are not harnessed to any particular normative goal. This shared logic of abstract proceduralism is not accidental, but exposes a deep connection between liberal political theory and the technological developments of late capitalism. Much has been written recently on both the dystopic and utopic aspects of contemporary technology, but less has been done to connect the underlying logics of liberalism with the functioning of technology itself.


In ``The Politics of Recognition'', Canadian philosopher Charles Taylor draws on Ronald Dworkin's essay on liberalism to distinguish between a ``procedural'' and a ``substantive'' form of liberalism. Procedural liberalism describes a society that ``adopts no particular substantive view about the ends of life'' but is rather ``united around a strong procedural commitment to treat people with equal respect'' \cite[56]{Taylor1994}: ``A liberal society must remain neutral on the good life, and restrict itself to ensuring that however they see things, citizens deal fairly with each other and the state deals equally with all'' \citep[57]{Taylor1994}. As is typical of Taylor's idealism, this conception of the liberal proceduralism is disconnected from the material reality of the changing moments of capitalism. In the current world of ``cybernetic'' or ``algorithmic'' capitalism \citep{DyerWitheford2015, Parisi2015}, for example, liberal proceduralism is not independent of developments in computing, data, and surveillance that make up what Gilles Deleuze has called the ``societies of control'' \citep{Deleuze1992}. For Deleuze, ``types of machines are easily matched with each type of society - not that machines are determining, but because they express those social forms capable of generating them and using them'' \citep[6]{Deleuze1992}, and he identifies the development of the computer with the passage from the society of discipline to the society of control. Deleuze comes close to Marx's famous description in The Poverty of Philosophy of the way in which ``in acquiring new productive forces men change their mode of production'' such that ``the hand-mill gives you society with the feudal lord; the steam-mill, society with the industrial capitalist'' \citep[95]{Marx1955}. But Deleuze avoids the determinism that this passage is often called upon to support by focusing on the political dynamics surrounding technological change, the transition from a society in which the individual confronts the mass to one which in which ``individuals have become `dividuals,' and masses, samples, data, markets, or `banks''' \citep[5]{Deleuze1992}. In this network of decentralized control or ``machinic  subjection'' \citep{Lazzarato2012}, Marx's vision of an automated capitalism comes to pass: ``Once adopted into the production process of capital, the means of labour passes through different metamorphoses, whose culmination is the machine, or rather, an automatic system of machinery{\ldots} set in motion by an automaton, a moving power that moves itself; this automaton consisting of numerous mechanical and intellectual organs, so that the workers themselves are cast merely as its conscious linkages'' \citep[692]{Marx1973}. Much attention has been paid recently to the quantification of the self \citep{Moore2017}, the data mining of ``platform capitalism'' \citep{Srnicek2016} and the development of a data-driven ``surveillance capitalism'' \citep{Zuboff2019}, but what I want to focus on here is the relationship between algorithmic capitalism and the liberal proceduralism described by Taylor, and to contrast both with the ``absolute procedure'' Antonio Negri identifies as a core element of radically democratic constituent power. 
	
The neoliberal transition that began in the early 1980s capitalized on the demands made by the generation of 1968 against the post-war consensus (social peace, respect for established institutions, repression of individual desire) to reinscribing an absolute individualism (in the figure of the consumer and the entrepreneur) at the heart of political theory. John Rawls' Theory of Justice set the tone by supposing that ``each individual has a rational plan of life drawn up subject to the conditions that confront him. This plan is designed to permit the harmonious satisfaction of his interests'' \citep[93]{Rawls2005} and argued that the role of liberal democratic institutions was to provide the neutral foundation for each individual to achieve their plan. However, the development of post-colonialism, Indigenous sovereignty, multiculturalism, and new social movements in the 1970s forced liberal theory to come to terms with questions of identity and community far from Rawls' individualist ``original position''. As Will Kymlicka put it in a 1989 article on ``Liberal Individualism and Liberal Neutrality'', ``one of the most persistent criticisms of Rawls theory of justice is that it is excessively individualistic, neglecting the way that individual values are formed in social contexts and pursued through communal attachments'' \citep{Kymlicka1989b}. The liberal-communitarian debate (really a debate within liberalism) attempted to import some of the lessons of Marxism, feminism, and various postmodernisms around identity and collectivity, but without jettisoning what was most characteristic of liberalism. Ronald Dworkin, in contrasting a ``value-neutral'' theory of equality - issuing directly from Rawls' understanding of individual rational plans of life - to a content-rich theory which included collective conceptions of the good, argued that it was precisely the neutrality of the liberal state towards any individual conception of the good that best protected individual rights and supported individual flourishing. Michael Sandel perceived the threat of this content-free neutrality in the rise of the procedural republic in which the alienation and isolation inherent in capitalist social relations were deeply inscribed within the state administration. Charles Taylor, in his investigation of Canadian multiculturalism, sought to balance the communitarianism advocated by Sandel with the respect for individual rights in a kind of ``moderate proceduralism''. As recent work such as Alan Patten's Equal Recognition has indicated \citep[patten-2014], the question of how to balance individualism with collective demands still hangs over liberal political thought.
	
For Antonio Negri, this question is unresolvable precisely because of the ways in which liberal institutions become static and closed. The algorithmic governmentality that is the end result of the procedural republic - made possible by the rapid development of ICTs and the availability of Big Data - is simply the latest example of the constituted power of capital and the state shutting down the democratic strength of the multitude. In this paper I will investigate Dworkin, Sandel, and Taylor's attempts to clarify the the positions of the individual and the collective in liberal theory, before turning to Negri's conception of constituent power.

	

\section*{Form, Content, and Procedure}


In his 1978 paper on liberalism, Ronald Dworkin´ sought to account for the dissolution of a ``substantive'' or ``constitutive'' agreement on what liberalism is. With the turn to neoliberalism, not only had liberalism become less clearly distinguishable from conservatism, but significant enough disagreements within liberalism led to an inability to agree on in its basic propositions. Given that different people can and do have different (substantive) opinions on what the good life consists in, Dworkin asks what it must mean for a government to treat its citizens as equals. The two possible answers Dworkin discusses are 1) that equality consists in the state remaining neutral with respect to any particular conception of the good life or 2) that this neutrality is impossible because the state ``cannot treat its citizens as equal human beings without a theory of what human beings ought to be'' \citep[127]{Dworkin1978}. These two opposing perspectives lie at the heart of many current social and political controversies, including the ``free speech crisis'' on North American campuses \citep{Mackinnon2018}, the risk that extending legal protection to transgender people further implicates them in structures of administrative discipline and oppression \citep{Spade2015}, and the BlackLivesMatter movement against police brutality and systemic racism. The non-liberal left tends to argue for the second theory of equality: that substantive difference must be recognized and accounted for, while both liberal and conservative positions tend to focus on the content-neutral (``colour-blind'') and procedural theory of equality.
	
For Dworkin, the first theory of equality is the one that leads to what are usually considered liberal values and political positions. The neutrality of the state with respect to any particular conception of the good prioritizes the content-neutral application of procedures to guarantee fair and equal distribution (of material good but also of opportunities):
	
\begin{quote}
Political democracy distributes opportunities, through the provisions of the civil and criminal law, as the citizens of a virtuous society wish it to be distributed, and that process will provide more scope for virtuous activity and less for vice than any less democratic technique. Democracy has a further advantage, moreover, that no other technique could have. It allows the community to use the processes of legislation to reaffirm, as a community, its public conception of virtue. \cite[137]{Dworkin1978}
\end{quote}

The two theories of equality Dworkin describes distinguish between an equality of content and an equality of form. The form/content distinction has been a philosophical problem going back to Plato and Aristotle, and is connected in Hegel's logic, to the problem of Existence and Appearance. Rather than being binary oppositions, Hegel's dialectical approach unites form and content into an essential reaction: ``Appearance is the dialectic of Form \& Content, or to put it another way, Form \& Content is the `essence of appearance'. It is bourgeois philosophy that divides form from content, in order for bourgeois science to operate on the form without having to worry about the content (we can see this in the description of mathematics as concerned purely with formal systems or ``formalisms''). To make reality tractable to (bourgeois) science and logic, form must be severed from content, made computable through the application of algorithms \footnote{we might usefully bear in mind the ``scientific management'' of Taylorism's time-motion studies}. Alan Turing, in his 1936 paper on computation, distinguished between the mechanical capacity of a machine to calculate or compute, as opposed to  human means: ``We may compare a man in the process of computing a real number to a machine which is only capable of a finite number of conditions'' \citep[231]{Turing1936} through which it must pass in order to calculate, step-by-step, a final result. The automated machine which can perform such a computation work on formal procedures, including numbers, without consideration of their content. Computability, Turing notes, is equivalent to Alonzo Church's ``effective calculability'', both of which differ from the intuitive and imaginative ability of the human mind to compute. The mind engaged in calculation does not separate form and content, it is not engaged in pure ``symbol manipulation'', but engages with both the formal processes and the semantic content in the process of thought itself.
	 	
The desire to reduce social and political questions to formal ones goes back at least to Hobbes if not to Plato, and Dworkin's content-free proceduralism belongs to that genealogy, but the algorithmic ordering of social, economic, and political processes in order to reduce risk and guarantee outcomes (if not the good life, then at least higher profits) has only become possible with the advent of new technologies, greater storage and memory capacities, and the developments of faster and more sophisticated algorithms, such as those employed in machine learning and other artificial intelligence or Big Data processes. Moore's Law - that computation power doubles roughly every two years - keeps the dream of an algorithmic discipline of society alive. Liberal proceduralism is the reflection of that dream. In his 2019 book on algorithmic governance, Ignas Kalpokas describes the ways in which digital data which works by establishing positivist correlations rather than meaningful causes and effects, ``can be subsequently worked on and turned into algorithmically devised courses of action, changes in the digital architecture of our everyday environment, or nudging strategies''. Echoing Marx, Kalpokas notes that ``this attitude that prides itself on replacing causes with trends also has the effect of altering the place of human persons, effectively objectifying and commodifying them, turning them into data generators where the data footprint is all that matters and is taken for the person'' \citep[2]{Kalpokas2019}. Statistical prediction \citep{Joque2019} and behavioural economics (i.e. ``nudging'') \citep{Bacevic2020}) have become extremely important in the context of the COVID-19 pandemic, but they simply extend a much older logic of instrumental rationality and social control in political thought. 
	

\section*{Sandel's ``Procedural Republic''}

The proceduralism vision of liberal society espoused by Dworkin was taken up by Michael Sandel in his 1984 paper on ``The Procedural Republic and the Unencumbered Self''. Sandel's article was an intervention into the liberal-communitarian debate which arose in the wake of John Rawls' Theory of Justice (1971), objecting to that work's ``unencumbered invidualism'' in a context of post-colonial immigration and multiculturalism, settler-colonial multinationality, and the new social movements that came out of the Civil Rights Movement and the upheavals of 1968. For Sandel, liberal proceduralism is inextricably linked to various conceptions of individual selfhood, in the first place the Kantian transcendental subject, which Sandel finds unconvincing: ``Kant's idealist metaphysic, for all its moral and political advantage, cedes too much to the transcendent, and wins for justice its primacy only by denying it its human situation'' \citep[85]{Sandel1984}. Sandel turns instead to look to an ``unencumbered self'' implied by Rawls' original position, for whom ``not the ends we choose but our capacity to choose them'' is most important, ``most essential to our personhood'' \citep[86]{Sandel1984}. This individualism imagines a purely free subject capable of purely free choice. ``Only if my identity is never tied to the aims and interests I may have at any moment,'' Sandel writes, ``can I think of myself as a free and independent agent, capable of choice'' \citep[86]{Sandel1984}. Thus, while

\begin{quote}

As unencumbered selves, we are of course free to join in voluntary association with others, and so are capable of community in the cooperative sense. What is denied to the unencumbered self is the possibility of membership in any community bound by moral ties antecedent to choice; he cannot belong to any community where the self itself could be at stake. \cite[86-87]{Sandel1984}

\end{quote}

The ``unencumbered'' individualism that informs Rawls' vision ends up being no more convincing than the Kantian transcendental subject. The critique of this individualism goes back to Marx's critique of Proudhon and the classical political economists, notably in The Poverty of Philosophy and in the 1857 Introduction to the Grundrisse. Where Marx and later social constructionists argue that the unencumbered self is a social impossibility (``Production by an isolated individual outside society{\ldots} is as much of an absurdity as is the development of language without individuals living together and talking to each other'' \cite[17]{Marx1973} - verify page), Sandel makes the weaker claim of undesirability:

\begin{quote}

To imagine a person incapable of constitutive attachments such as these is not to conceive an ideally free and rational agent, but to imagine a person wholly without character, without moral depth. For to have character is to know that I move in a history I neither summon nor command, which carries consequences nonetheless for my choices and conduct. It draws me closer to some and more distant from others; it makes some aims more appropriate, others less so. As a self-interpreting being, I am able to reflect on my history and in this sense to distance myself from it, but the distance is always precarious and provisional, the point of reflection never finally secured outside the history itself. But the liberal ethic puts the self beyond the reach of its experience, beyond deliberation and reflection. Denied the expansive self-understandings that could shape a common life, the liberal self is left to lurch between detachment on the one hand, and entanglement on the other. Such is the fate of the unencumbered self, and its liberating promise. \citep[90-91]{Sandel1984}

\end{quote}

Writing in the early 1980s, Sandel is unable to see what has become clear in hindsight: the dismantling of the Welfare State and the neoliberal instrumentalization of politics that Foucault called ``governmentality''. Sandel glimpses the problem, but his vision of 1980s America as continuous with, rather than ruptured from, the US of the New Deal and the Great Society makes him unable to fully understand the problem: 

\begin{quote}

Notwithstanding the extension of the franchise and the expansion on individual rights and entitlements in recent decades, there is a widespread sense that, individually and collectively, our control over the forces that govern our lives is receding rather than increasing. This sense is deepened by what appear simultaneously as the power and the powerlessness of the nation-state. One [sic] the one hand, increasing numbers of citizens view the state as an overly intrusive presence, more likely to frustrate their purposes than advance them. And yet, despite its unprecedented role in the economy and society, the modern state seems itself disempowered, unable effectively to control the domestic economy, to respond to persisting social ills, or to work America's will in the world. \citep[92]{Sandel1984}.

\end{quote}

In hindsight, we can recognize the ``procedural republic'' identified by Sandel more precisely as the ``algorithmic republic'' of computer-driven neoliberal social, political, and financial logics. Sandel's misidentification of what was going on makes him see the problem with the procedural republic as the anti-liberal one of the concentration of power: ``liberty in the procedural republic is defined in opposition to democracy, as an individual's guarantee against what the majority might will. I am free insofar as I am the bearer of rights, where rights are trumps. Unlike the liberty of the early republic, the modern version permits - in fact even requires - concentrated power. This has to do with the universalizing logic of rights. Insofar as I have a right, whether to free speech or a minimum income, its provision cannot be left to the vagaries of local preferences but must be assured at the most comprehensive level of political association. It cannot be one thing in New York and another in Alabama.'' \cite[93-95]{Sandel1984}. This universalization, however, is not a function of the concentration of power in human hands, but the making tractable, making computable, making predictable the unruly, messy data of human life for algorithmic procedure. Sandel misrecognizes the formalist imperative in algorithmic capitalism and neoliberal governmentality. When Sandel argues that ``in our public life, we are more entangles, but less attached, than ever before'' and that ``it is as though the unencumbered self presupposed by the liberal ethic had begun to come true - less liberated than disempowered, entangled in a network of obligations and involvements unassociated with any act of will'', he, like Dworkin, does not understand that this is not a mistake or an aberration, but a reflection of the real socio-economic developments of post-welfare State capitalism. Where Dworkin divides theories of equality along form/content lines - thus inscribing the mathematical formalism of computability into his political theory - Sandel sees the statistical flattening of individual social relations and histories as an authoritarian tendency at odds with the liberal tradition.  Neither Dworkin's nor Sandel's liberal theory can comprehend the ways in which liberal thought as such is subject to capitalist (technical/instrumental) logics and capitalist restructuring. Sandel's aporia arises from his identification of individualistic liberalism - which in principle should be opposed to the concentration of power - and democracy, rather than seeing liberalism as the hegemonic ideology of of capital itself. 


\section*{Charles Taylor and the Politics of Recognition}

Charles Taylor, in his investigations of multiculturalism and cultural difference, follows Sandel in his recognition that the pure individualism of Rawls - enshrined in the primacy of individual rights - would, in a fully developed procedural republic, lead to the erasure of all difference. But Taylor believes that, far from being completely antagonistic, a society based on Dworkin's first theory of equality (which, for Taylor, means English Canada) can work out a liberal coexistence with a society based on the second (in Taylor's view, Quebec). In this sense, then, the Taylor tries to reconcile the individualistic-communitarian contradiction, as well as Dworkin's form/content antinomy \footnote{like Will Kymlicka, Taylor argues that some individual rights must be sacrosanct, while others can be balanced against communal needs. For Kymlicka (1995) polyethnic multiculturalism (i.e. of immigrant groups) would be able to be accommodated, while multinational multiculturalism (like Quebecois or Indigenous sovereignty) would not}.
	
Taylor believes that the flattening out of the procedural republic, the making uniform of the richness of life to make it tractable to algorithmic capital, can simply be resisted by the application of the correct constitutional theory. Such a theory would mix Dworkin's two theories of equality. Thus, while in such a society ``there would be no question of cultural differences determining the application of habeas corpus, for example'', a ``broad range of immunities and presumptions of uniform treatment that have sprung up in modern cultures'' \citep[61]{Taylor1994}. Resistance to the flattening-out uniformity of neoliberalism would simply be a matter of ``weigh[ing] the importance of certain forms of uniform treatment against the importance of cultural survival, and opt sometimes in favour of the latter''. The existence of multiculturalism in a post-neoliberal conjuncture indicates, for Taylor, that ``the rigidities of procedural liberalism may rapidly become impractical'' \citep[61]{Taylor1994}. However, what we are seeing today - for example in the rapid development of facial recognition artificial intelligence technology in conjunction with the demands for a content-specific equality on the part of people of colour - is that the algorithmic republic meets the threat posed by difference with more computation, more statistical probability, more prediction. As a result, demands for the protection of difference leads to yet more uniformity and the inscription of marginalized subjects deeper and deeper into an algorithmically administrative structure of laws and the police. The combination of trans rights and prison abolitionism in, for example, the work of Dean Spade, indicates the fundamental incompatibility of difference with state administrative institutions. The politics of recognition held out by Taylor is proving (has proven) to be inadequate to the contradiction between difference and computability, content and form, which lies at the heart of the attempts - by Dworkin, Sandel, Taylor, and many others - to salvage liberal theory in the face of rapidly worsening capitalist crisis. We must look elsewhere for a theory of society and power which might overcome the false dichotomy, the binary logics, of capitalist technology reflected in liberal thought. 


\section*{Constituent Power as ``Absolute Procedure''}

In the earliest days of the rapid expansion of information and communications technologies (ICTs), Antonio Negri saw them as holding a communicative and collaborative potential to adapt to the neoliberal expansion of the subsumption of labour under capital. The development of ICTs, he argued, was critical to the movement from the mass to the socialized worker. Drawing on the tension identified by Italian workerists and autonomists between capital's need to foster increased collaboration and cooperative work, and the radical effects this cooperation had on worker solidarity and directed action, Negri saw the development of technology under neoliberalism as at the same time erasing and flattening out difference and as providing the basis for ``a new class composition and a new political subject'' \citep[48]{Negri1989}. Negri saw the reality of the neoliberal transition and the necessary response more clearly than Dworkin, Sandel, or Taylor, and while his utopian view of the potential of ICTs has often been dismissed, along with the theory of ``cognitive capitalism'' as naive ``accelerationism'', the political theory that derived from his understanding of neoliberalism and technology more sophisticated than that, and is important to the question of identity and communal society. 
	
The uniformity Dworkin, Sandel, and Taylor saw darkly Negri recognized as a situation in which ``production and reproduction constitute a completely uniform, undifferentiated network'' \citep[89]{Negri1989}. The technological advances of neoliberalism depended on the flattening-out, the making average, of the workforce \footnote{Thus fulfilling Marx's analytical insight that the labour is valued as a social average}: ``Society thus offers itself to work in the same way that, in the factory, a single machine, several machines, and the entire system of machinery offered themselves to the labour force: namely, as a system of preconditions{\ldots} The machines constitute a system because this world of technical conditions is an ordered universe and a sort of ideal schema to which new activity can and must be added'' \citep[90]{Negri1989}. This ordered universe is the only kind in which algorithms can predict correctly and avoid risk. For Negri, this ordering has severe repercussions for Marx's theory of value (though Negri's interpretation of the labour theory of value under neoliberalism has been contested, for example in \cite{caffentzis-letters}). The flattening out of difference by algorithmic capital is part of the machinic ecology of contemporary capitalism, and cannot be reformed by tweaking liberal political theory. Rather, the insistence on difference, for Negri, lies at the heart of the class struggle: ``diversity and antagonism represent a choice in favour of the values of life and the quality of reproduction, and a rejection of negative values, of the practices of death and the nullifying tendencies that are implicit in the capitalist machine'' \citep[94]{Negri1989}.
	
Where the procedural republic was, for Sandel, a violation of the principles of liberal communitarianism, Negri sees it as the culmination of the process of subsumption from the factory to society, of the transformation of the mass workers into the socialized worker, with which liberal thought has been complicit. The institutions that Dworkin and Taylor in particular look to to salvage liberal individualism are already compromised, in Negri's view, because they, like the factory and like capital itself, form the constituted power that is by definition antagonistic to any of what Sandel calls the ``constitutive attachments'' that make life worth living. Indeed, the distinction drawn by Dworkin between two antagonistic theories of equality, a distinction Sandel and Taylor can only moderate but not overcome, only makes sense within the context of the constituted power of liberal politics (law, rights, representative democracy, elections etc). For Negri, the choice is a false one, as is the antagonism between individual and collective life that is another obstacle for Sandel, which Taylor seeks to overcome in his politics of recognition. 
	
For Negri, the contradiction between the irreducible differences and rich constitutive attachments of human life on the one hand and the constitution of an ordered society on the other is a problem the political theory tries in vain to resolve. Because constitutional theory derives its legitimacy from ``the people'', it always needs to find a way to limit or put an end to the constitution-forming activity of the multitude. Once the people (in the form, say, of a constituent assembly) has done its work and a new juridical and political order has been constituted, it must then be disbanded and its power made subject to the legal order it has legitimated. The question for traditional political theory, as Negri conceives it, is as follows: 
	
	\begin{quote}
	How can we keep open the source of the vitality of the system while controlling it? Constitutuent power must somehow be maintained in order to avoid the possibility that its elimination might nullify the very meaning of the juridical system and the democratic relation that must characterize its horizon. \cite[4]{Negri1999}.
	\end{quote}

If the constituted order - including the procedural republic - owes its legitimacy to ``the power of the people'', then that power cannot merely be dispensed with once the legal-political institutions are in place. It must be maintained somehow, even if only discursively, as a source of legitimacy of the system itself. However, the constituted power of political institutions must always try to overcome or neutralize the threat posed by constituent power itself. As Negri puts it, ``the relationship that juridical theory (and through it the constituted arrangement) wants to impose on constituent power works in the direction of neutralization, mystification, or, really, the attribution of senselessness'' \cite[10]{Negri1999}. If we equate Dworkin's second theory of equality, in which the government holds to a substantive vision of the good, then we might be tempted to argue that the procedural theory is the theory of constituted power, and the substantive theory that of constituent power. But Negri refuses this false dichotomy completely: Dworkin's model assumes a division between constituted and constituent power in both theories of equality. The government is divided from the people (despite owing its legitimacy to the people), the government is an administrative apparatus rather than the expression of the collective decision of the multitude as such. Negri's vision of constituent power is that of an absolute democracy, and democracy as an absolute power.
	
Procedural liberalism, especially in its current, algorithmic form, seeks to replace the absolute democratic power of the multitude by the once-and-for-all rules of algorithmic logic. It seeks to institute the strength of the multitude in process and the rule of law that are decided on and then obeyed. For Negri, however, 
	
	\begin{quote}
	at the very moment when strength gets instituted it ceases being strength and thus declares itself as never having been such. There is only one correct (and paradoxical) condition for a definition of sovereignty linked to that of constituent power: that it exists as the praxis of a constitutive act, renewed in freedom, organized in the continuity of a free praxis. But this contradicts the entire tradition of the concept of sovereignty and all its possible meanings. Consequently, the concept of sovereignty and that of constitutive power stand in absolute opposition. \cite[22]{Negri1999}
	\end{quote}

The ``constitutive attachments'' which Sandel sees threatened by the procedural republic are, in Negri's view, never truly threatened, precisely because real strength derives from the multitude. Constituent power does not seek institutionalization, but seeks only ``more being - ethical being, social being, community'', and it is in this way that constituent power is the real form of democracy. This resistance to institutionalization means that all the rules, procedures, algorithms, and processes of constituted power (of the state, of capital, of capitalist technology) can never dominate over the ongoing decision-making of the multitude. Those rules, procedures, etc, are there to avoid decision-making, to remove decision from the ambit of the multitude and to place it within the static confines of constituted power.
	
This power of collective, open-ended decision-making Negri calls ``absolute procedure'', the production of the forms-of-life that make community and communal being possible: ``The political is here production, production par excellence, collective and non-teleological'' \citep[27]{Negri1999}.
	
For Negri, the past participle (``constituted'') indicates a final closure of the constituent power of the multitude. It is in this respect that ``absolute procedure'' differs from liberal proceduralism - the constituent procedure is absolute because it is unfinalized and unfinalizable, while liberal procedure is closed and finds its perfection in static form of the algorithm. The trade off of the perpetual openness of constituent power is risk, risk that is intolerable to the capitalist quest for growth and profit: ``The constituent principle is not a dialectical principle - it cannot be resolved or overcome - but this quality itself keeps it in a terribly precarious state'' \citep[60]{Negri1999}. The ongoing crisis of constituent power is the tendency to constitutedness, to finalization, institutionalization, algorithmization: ``the constituent power and strength are in fact absolute, but any actualization opposes them, wants to deny their absoluteness. If the absolute overflows or dislocates, it finds itself confronted by the rigidity of the constituted'' \citep[60]{Negri1999}. 
	
The neverending process of exercising strength and making decisions is, however, the heart of democracy for Negri. Paradoxically, therefore, it is constituent power rather than liberalism that can guarantee the equal treatment of difference; not a value-free or content-neutral equality, but an equality that sees difference as richness. Difference is a valued production of the multitude, rather than a deviation to be stamped out in the name of uniformity, or to be vaguely tolerated as long as it poses no threat to liberal procedure and individual rights. Thus, as Negri remarks, ``absoluteness never becomes totalitarianism'', the strength of both individuals and the collective ``is expressed and nourished by discord and struggle{\ldots} and the construction of the political is the product of permanent innovation'' \citep[29]{Negri1999}. The algorithm - the procedures of the liberal republic - represent closures, the shutting down of individual agency and collective power. Constituent power is, on the contrary, always open: ``it is at the same time resistance to oppression and construction of community; it is political discussion and tolerance; it is popular armament and the affirmation of principles through democratic invention'' \cite[29]{Negri1999}.
	
Negri thus rejection the opposition between Dworkin's two theories of equality, between Sandel's procedural republic and the constitutive attachments of human life, between Taylor's sacrosanct individual rights and the balance of communitarianism. These choices are false ones, set up by liberalism's position as role mouthpiece of capitalist social, economic, political, and technological relationships. Every attempt to strike the ``right'' balance between competing worldviews, or between the individual and the collective, or between universalism and particularity, is doomed to fail precisely because they are conceived under the aspect of the constituted power of the capitalist state. If we reject this aspect, then those political problems fall away.


\section{Homoiconicity and the Imitation Game}

The traditional algorithms used in computer programming are seen as pure formalisms different from the data they operate on, with unchanging rules and orders of operation. They are closed, final, static, and run in a neutral fashion, irrespective of the semantic content of the data. In this, programming algorithms are mirrors of liberal procedures. However, if Turing in 1936 proposed a model for a universal machine that would operate properly on any instruction set, by 1950 he was developing a new model which treated form and content, algorithm and data, not as two isolated entities, but as a single changing process. For Turing, artificial intelligence in the form of learning machines exemplified, like Negri's constituent power, an ``absolute procedure''.
	
In the paper ``Computing Machinery and Intelligence'' which introduced the ``imitation game'' and set the stage for the artificial intelligence research of the later 1950s, Turing  argued against an algorithmic view of human behaviour: ``It is not possible to produce a set of rules purporting to describe what a man should do in every conceivable set of circumstances'' \citep[452]{Turing1950}. If machinery is to have the capacity of at least imitating human behaviour, the fixed, static nature of algorithms must be discarded, a process Turing equates with learning itself: 

\begin{quote}
The idea of a learning machine may appear paradoxical to some readers. How can the rules of operation of the machine change? hey should describe completely how the machine will react whatever its history might be, whatever changes it might undergo. The rules are thus quite time-invariant. This is quite true. The explanation of the paradox is that the rules which get changed in the learning process are of a rather less pretentious kind, claiming only an ephemeral validity.\cite[458]{Turing1950}
	\end{quote}

While Turing sees intelligent behaviour as consisting precisely in ``a departure from the completely disciplined behaviour involved in computation'', he nonetheless sees this departure as only a very slight one. If we turn this process around, we could say that rather than a \textit{departure} from the fixed order of a given algorithm, a departure from the existing rules, learning and intelligent behaviour is in fact a \textit{production }of new rules, new algorithms. This is the central element in Negri's idea of an absolute procedure, a procedure that is not bound by any existing rules, and which does not \textit{depart} from rules but is constantly producing new rules. 
	
How does this changing of the rules/algorithm work, either for Turing or for Negri. The answer lies in not drawing any kind of distinction between form and content, code and data, political procedure and lived experience. In computer programming, the phenomenon of code and data being indistinguishable is called \textit{homoiconicity}.
	
The LISP language (now family of languages) which was developed by John McCarthy in 1956 in the aftermath of Turing's paper, exhibits homoiconicity: its code and its data take the same form. Code can be used as data and data can be used as code, with the result that a program can re-write itself, can change its own algorithm as it is running. If we have, on the one hand, a political theory that sees political procedure and the multitude as indistinguishable elements of a single political subject, and on the other hand, a technological model that overcomes the binary division between form and content, could we not ask what political situation would have to be the case for this polity and this technology to have the dominance that Silicon Valley and liberal proceduralism have now? Such a transcendental question might open new space for an exploration of a radical democratic mode of production. 

	
\section{Conclusion}

The distinction between procedure and content - the rule of law and lived experience - is a false one shared by both contemporary technology and liberal political thought. Computer programming and liberal theory are both reflections of real changes in the material relations of production dating back to the Second World War and achieving a particular dominance with the neoliberal transition. Attempts to reconcile procedure and content, in Taylor's politics of recognition, for example, cannot be successful precisely because liberal thought reflects the realities of social relationships. The radical democracy of Negri's constituent power, on the other hand, precisely because it is an ``absolute procedure'' intimately connected with the flourishing of the multitude, avoids the false distinction between juridical form and the content of life.

\backmatter

\bibliographystyle{plainnat} 
\bibliography{Bibliography}
\end{document}