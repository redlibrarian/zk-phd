% !TEX encoding = UTF-8 Unicode
\documentclass[12pt,oneside]{memoir}
\usepackage{geometry}
%\geometry{letterpaper}
\geometry{a4paper}
\usepackage{graphicx}

%% For highlight ranges of text marked with inspector comments
\usepackage{soul}

\usepackage{xcolor}
\usepackage[
    colorlinks=true,
    urlcolor=blue,
    linkcolor=blue,
    citecolor=blue,
    filecolor=blue,
]{hyperref}
\usepackage{memhfixc}
\usepackage{natbib}
\setcitestyle{authoryear,open={(},close={)}}


\setcounter{page}{1}
\pagenumbering{roman}

\title{Constituent Power and the Politics of Recognition: The Case of Intellectual Freedom in Canadian Libraries}
\author{Sam Popowich}
\date{2020-06-30}

\usepackage{palatino}

\begin{document}

\maketitle
\clearpage


\tableofcontents

% \listoffigures
% \listoftables

\newpage
\setcounter{page}{1}
\pagenumbering{arabic}

\mainmatter

\chapter{Introduction: Resurgence of the People}
\label{scrivauto:5}

\section{The Politics of Recognition: Cultural Diversity and Identity Politics}
\label{scrivauto:6}

The theory of the ``politics of recognition'' arose as a particular tendency out of the communitarian/liberalism debates of the 1980s. The turn to neoliberalism saw the beginnings of the dismantling of the Welfare State along with a decline in the social solidarity of the post-war consensus. This decline was in part caused by, in part opened the door to, expressions of individual desire and a rejection of communitarian compromise in the name of consumer choice on the one hand, and the radical rejection of the Establishment in the counterculture of the 1960s, whose most radical expression was in the worker and student revolts of 1968. This resurgent individualism was harnessed by neoliberal theorists in their push to implement free-market reforms and to drive individual consumerism as two of the main [drivers] of their political and economic theory \citep{harvey-2005}. The social and political effects of this transition are diagnosed by, for example, Marcuse, but also by Deleuze, Guattari, Derrida, and others: the shift to neoliberalism can, following Jameson, be understood as provoking a shift towards post-structuralism and other ``postmodern'' philosophical positions \citep{Jameson1991}. Justice - both individual and post-colonial - played a major role in the development of post-structuralist theory at the end of the 1960s. The editors of \textit{Deconstruction and the Possibility of Justice }note that while ``at least by its critics, deconstruction has been associated with cynicism towards the very idea of justice'', it is ``in some way, aligned with the marginalized'' \citep[ix]{CornellRosenfeldCarleson1992}.
In response to this, liberal theorists like John Rawls sought to restore liberalism's primacy in questions of justice and political philosophy, with Rawls' \textit{Theory of Justice }(1971) ``reinvigorat[ing] `high liberalism' \citep[3]{Galisanka2019} for the neoliberal turn. By the end of the 1970s, however, as neoliberalism had led not only to the political projects of Thatcher, Reagan, and others, a reinvigoration not just of liberalism, but libertarianism, had developed. Robert Nozick's \textit{Anarchy, State, and Utopia} (1974) directly challenged Rawls' conception of the role of the state in the distribution of justice, and argued. Connecting Nozick's individualist libertarianism and a ``methodological individualism'' present in Rawls' \textit{Theory} with the effects of the neoliberal dismantling of the Welfare State, especially after the elections of Thatcher (1979) and Reagan (1980), many political philosophers such as Michael Sandel, Michael Walzer, Alistair McIntyre, and  Charles Taylor began to question the validity of purely individualistic political theory, and attempted to counter the ``atomism'' \citep{Taylor1985} of neoliberal politics with a more communitarian approach.
These issues took on particular resonance in the context of specific political struggles and controversies in Canada - over multiculturalism, Quebec sovereignty, and Settler-Indigenous relationships in particular - and led to a particularly Canadian ``politics of recognition'' in the 1990s. We will look at that development in a moment, but first it is useful to place the communitarian-liberalism debate in the broader context of political thought.
In a major work on the politics of recognition, Axel Honneth traces a decline of situating individuals first and foremost in their social and collective relationships to the advent of modernity and the development of capitalism. The social contract theory of, for example, Hobbes - underpinned by the development of the ``bourgeois ideology'' of Descartes and the scientific revolution - inaugurated the methodological individualism that would become central to liberal political philosophy from Locke onward. Hegel, on the other hand, challenged such individualistic social theories (what Marx dismissed as ``Robinsonades'') in the name of the primacy of social relations. In Honneth's words, 

\begin{quote}

Hegel labels all those approaches to natural law [e.g. Hobbes'] `empirical' that start out from a fictitious or anthropological characterization of human nature and then, on the basis of this and with the help of further assumptions, propose a rational organization of collective life within society. The atomistic premises of theories of this type are reflected in the fact that they always conceive of the purportedly `natural' form of human behaviour exclusively as the isolated acts of solitary individuals, to which forms of community-formation must then be added as a further thought, as if externally. \citep[12]{honneth-struggle}. 

\end{quote}

This critique of social contract theory, in which individuals come first and then choose or decide to come together in a community or society, is Marx's main critique of Proudhon in \textit{The Poverty of Philosophy}, and in the 1857 `Introduction', Marx argues that the isolated individual which liberal political economists take as their starting point is in fact the \textit{end result} of a process of social and political alienation which took place over hundreds of years. The transhistorical timelessness and universality of liberal political thought is, as we will see, unable to understand the historical (that is, changing) nature of subjectivity itself, tends to misread Hegel's intersubjective conception of identity-formation, and ignores the entire tendency of Marxist engagement with the question of the relationship of individuals to society. As a result, liberal philosophers like Taylor make the same mistake as the classical political economists: they presume the isolated, atomistic individualism of capitalist modernity to be a timeless truth about human nature (rather than the result of specific social and political processes), and therefore take individuals as the social starting point, even as they offer a communitarian critique of both ``high liberalism'' and libertarianism.

In Canada, a practical politics of recognition came before its theoretical formulation in the aftermath of the communitarian/liberalism debate. In 1969, the Canadian Government attempted to leverage the post-war sense of social solidarity and national unity to impose a policy of assimilation on Indigenous peoples (embodied in the 1969 ``White Paper''). Indigenous resistance to this policy was strong, and the government quickly backed away from such a project, preferring instead to adopt ``recognition'' as the primary way of negotiating relations between Indigenous peoples and the settler-colonial state. This shift must be understood in the context of other international decolonizing and post-colonial struggles (for example in Algeria), and in this sense is connected with Quebec separatism which reached a peak in the October Crisis of 1970. While the defeat of the FLQ marked the end of support for violent insurrection in the name of Quebecois independence, it set the stage for Canadian multicultural debates (themselves necessitated by the increase in immigration in the aftermath of decolonization) and, after the patriation of the constitution in 1982, for several rounds of unsuccessful attempts at constitutional reform. These attempts brought about a crisis in the method of recognition, as the recognition of Quebec as a distinct society came up against the impossibility of recognizing Indigenous self-determination, and recognition was reduced to an affirmation of ``cultural'' identity (enshrined in the Multiculturalism Act of 1988) rather than a means of social, political, and economic transformation (``affirmation'' and ``transformation'' are used here in Nancy Fraser's sense).

The theory of the politics of recognition grew out of the communitarian critique of individualistic liberalism, attempting to restore community and social relationships to a place in their social ontology, if only a subordinate one. The politics of recognition in Canada - exemplified by the work of Taylor and Tully - reinforced the ``empirical'' perspective of social contract theory, in which individuals come first and social relationships are added after the fact. Despite relying on Hegel's master/slave dialectic in his essay on recognition \citep{Taylor1992}, Taylor still sees the intersubjective recognition as being entered into by autonomous individuals. In addition, the reduction of recognition to ``cultural recognition'' (as in Tully) constrained such a politics to an affirmative rather than a transformational order.
The prior existence and causal power of social relations, on the one hand, and the rejection of redistribution in favour of recognition on the other hand, led fairly quickly to critiques of recognition from the perspectives of Indigenous rights, minority rights, and identity politics. At the same time, recognition became and continues to be one of the major demands of marginalized communities because it includes them in the state and juridical apparatus of the protection of rights (see, for example, Sally Hines' work on recognition and the UK Gender Recognition Act of 2004 \citep{Hines2013}). Nancy Fraser has criticized recognition for its restriction to an ``affirmative'' role in social justice, leaving the structures of oppression and injustice intact \citep{fraser-justice, fraser-honneth}. Glen Sean Coulthard has criticized Taylor in particular for assuming an equality between the actors in intersubjective recognition, which erases the vast power differential between the settler state and Indigenous peoples \citep{coulthard2014}. Dean Spade has challenged the value of recognition and inclusion from the perspective of transgender people because recognition paradoxically inscribed them more deeply into structures of state power, police brutality, and the administrative violence of gender norms \citep{Spade2015}. Jakeet Singh has criticized recognition for its `top-down' approach to rights \citep{singh-recognition} and calls for a decolonized radical democracy as an alternative approach \citep{singh-democracy}.Thus, while recognition continues to be an important tool in the construction and maintenance of liberal-democratic hegemony over marginalized peoples, its insistence on an individualistic social-contract approach to identity formation on the one hand, and a cultural idealism which resists material transformation on the other, leads the politics of recognition continually into the kind of aporias we see playing out in controversies within librarianship (to which we will return).

\section{Hegel and the Ethical Unification of the People}
\label{scrivauto:7}

In many ways, the communitarian/liberalism debate simply reproduced Hegel's critique of atomistic individual social theory from the vantage point of the transition from the post-war compromise to neoliberalism. Communitarians like Taylor, Kymlicka, and to some extent James Tully, recognized the corrosion of society brought about by the ``atomism'' of individualistic neoliberalism, as well as the challenges to social solidarity necessitated by the struggle of social movements based on allegiances to collectivities different from the nation-state, and so their goal was, like Hegel's, the achievement of a ``natural ethical life'' (\citep[102]{Hegel1979}) arising out of a ``genuinely free community of living connections'' (\citep[145]{Hegel1977}). Like Hegel's opponents (mainly Kant and Fichte), the communitarian critique of liberalism attempted to show that a natural ethical life could not be predicated on an individualistic social ontology. However, even the communitarian perspective that informs Taylor's politics of recognition, Kymlicka's multicultural citizenship, and Tully's politics of cultural recognition, do not go far enough. They continue to be based on an atomistic individualism, on ``the existence of subjects who are isolated from each other'' and therefore the communitarian perspective also cannot form the basis of a ``condition of ethical unification among people'' because the necessary social and communal relationships have to be added to them after the fact (\citep[12]{honneth-struggle}).
I will claim that this mistake is based on a misreading of Hegel on the part of Taylor, who ignores Hegel's early work on political theory to base his interpretation of Hegel's theory of recognition solely on a short passage from \textit{The Phenomenology of Spirit}. Taylor ignores or misunderstand three key aspects of Hegel's political ontology. The first, as we have seen, is that individual subjects are born into social relations. As Honneth puts it, for Hegel ``every philosophical theory of society must proceed not from the acts of isolated subjects ut rather from the framework of ethical bonds, within which subjects always already move'' (\citep[14]{honneth-struggle}). The second is that individual identity is static and unchanging, so that the process of recognition becomes no longer (as in Hegel) a dynamic process of change, but merely the social contract rephrased in Hegelian terms. Third is that the subjects involved in the process of recognition are equals. Both the static nature of identity and the presumption of equality (i.e. the evacuation of power differences) is characteristic of the bourgeois ideology that informs liberalism in both its individualistic and communitarian forms.
These three divergences from Hegel's political philosophy in Taylor's theory of identity and recognition go on to inform the constitutional theories of Kymlicka and Tully. In Chapter Two I will dig more deeply into Taylor's theory of the self, community, and recognition, and then show, first, how Honneth's more sophisticated reading of Hegel undercuts Taylor's theory, placing Kymlicka and Tully's constitutional theories in jeopardy, and second, how Hegel's political theory in fact has more in common with Antonio Negri's theory of constituent power than is usually allowed. For example, while for Hegel atomistic social ontologies can only conceive of a community ``on the abstract model of a `unified many', that is, as a cluster of single subjects, and thus not on the model of an ethical unity''(\citep[12]{honneth-struggle}). While corporate social and political theories try to forge a community under the banner of a single collectivity (nation, or people, for example), the \textit{multitudo }is, in the words of Paolo Virno , ``the form of social and political existence for the many, seen as being many'' ``without converging into a One'' (\citep[21]{Virno2004}). The multitude, for Negri and Virno, is both a ``many'' composed of individuals, while also being an ``ethical unity'' of the type envisaged by Hegel.
For Hardt and Negri, the multitude preserves the differences of individual subjects, without forcing them into a unity (``people'' or ``masses'' or even ``working class''), and yet managing  ``to communicate and act in common while remaining internally different'' (\citep[xiv]{HardtNegri2004}). ``Insofar as the multitude is neither an identity,'' they write, ``nor uniform{\ldots} the internal difference of the multitude must discover \textit{the common} that allows them to communicate and act together'' (citep[xv]{HardtNegri2004}). In Chapter Three we will compare Kymlicka and Tully's constitutionalism against the ``ethical unification of people'' and (in Chapter Four) demonstrate how Negri's theory of the multitude both avoids the pitfalls of the liberal social ontology while also answering the criticisms of the politics of recognition made by Indigenous, Feminist, and Queer critics.

\section{The Multicultural Context of the Canadian Politics of Recognition}
\label{scrivauto:8}

In Canada, the political theory of recognition, minority, and group rights played a significant role in real political debates which have been taking place since the 1970s, but especially in the period of Indigenous resurgence and Quebec nationalism in the early to mid 1990s. As we have seen, the neoliberal turn of the 1970s disposed of the post-war consensus which papered over racial, gender, and class antagonisms in the name of social solidarity. By the late 1960s the question of ethnic minorities in particular returned with a vengeance, in the context of wars of liberation (Algeria, Vietnam), immigration (Enoch Powell's ``Rivers of Blood'' speech in 1968), and decolonization. The interwar period had seen the development of a system of bilateral treaties designed to protect ethnic minority rights and enable the the ``national self-determination'' enshrined in Woodrow Wilson's Fourteen Points. As \cite[2]{kymlicka-1995} points out, by 1939 this system had become unworkable, and Hitler was able to invade Poland and Czechoslovakia and annex Austria on the basis of the self-determination of ethnic Germans. The anti-imperial struggles of the 1940s-1960s (India, Algeria, Congo, etc) dovetailed with struggles for individual self-determination in the face of out-of-date establishment cultures in Europe and America. For example, the May 1968 uprisings in France were formed by a combination of new left-wing agitation (students in addition to workers), expressions of radical individual self-determination, and post-colonial immigrant experience, though the relationship between these three dynamics is not as simple as was once believed. As Maud Anne Bracke writes,
\begin{quote}
	``'1968', this, meant France's definite shift to postcoloniality. While de-colonization had started earlier, it was only in 1968 and its aftermath that French society became aware of the *permanent* presence of post-colonial immigrants. `1968' was the opening of the Pandora's box that contained the complex, explosive cluster of problems related to multicultural society. With their contradictory attitudes, the new left and the student movements in 1968 prefigured the failure of French society and the state in the decades to come, to engage with postcolonial immigrants as at once full and equal members of society and communities with distinct cultures and identities.'' \citep[128]{bracke-2009}
\end{quote}

David Harvey, in his Brief History of Neoliberalism, marks 1968 not only as the introduction of  postcolonial immigrant multiculturalism in to Western political life, but of the irruption of an individualistic conatus after two-decades of a post-war consensus that required the repression of individual desires for the collective good of the welfare state: ``The student movements that swept the world in 1968{\ldots} were in part animated by the quest for greater freedoms of speech and of personal choice. More generally, these ideals appeal to anyone who values the ability to make decisions for themselves.'' \citep[5]{harvey-2005}

Harvey goes on to argue that the explosion of the desire for individual freedom opened the door to neoliberal policies based on radical individualism and free-market fundamentalism, deregulation and privatization, summed up by Margaret Thatcher's remark that ``there's no such thing as society, there are individual men and women and there are families''. As Stuart Hall remarked about this conjunctural shift, while ``the Keynsian welfare state tried to set `the common good' above profitability', neoliberalism meant that `the function of the liberal state should be limited to safeguarding the the conditions in which profitable competition can be pursued without engendering Hobbes' `war of all against all''' \citep[707]{hall-2011}

It was in this context that the 1969 ``Statement of the Government of Canada on Indian policy'' (known as the White Paper) was released. The White Paper called for the elimination of ``Indian Status'', the abolition of the reserve system, and the complete assimilation of Indigenous peoples. In the context of late-1960s post-colonial politics, the White Paper drew widespread condemnation and was withdrawn in 1970. What replaced the assimilationist policy of the Canadian Government, according to Glen Sean Coulthard, was a ``politics of recognition'' enshrined in the Calder v. British Columbia case of 1973. Calder v. British Columbia was a Supreme Court case which recognized for the first time aboriginal title to land prior to colonization. The case had a profound effect on Indigenous land claims and marked a watershed moment in the development of ``recognition'' as a mechanism within Canadian politics. 

Additionally, the Canadian government was proud of the multicultural policy it adopted in 1971, there was pushback against it: ``The 1990s saw the manifestation of strong opposition to multiculturalism in certain quarters of the public and political spheres, and the drastic shrinking of bureaucratic structures devoted to it. In hindsight, the earlier opposition to multiculturalism by prominent journalists can be viewed as being at least a harbinger of - if not a spur to - the growth in antagonism towards the policy \citep[440]{karim2002}. [[karim-note]] The work on the politics of recognition in the early 1990s was an attempt to save or recuperate the idea of multiculturalism, to put it on a firm philosophical and political foundation.

Following the patriation of the Canadian Constitution in 1982, a round of (failed) constitutional amendments were proposed, the most contentious of which was the recognition of Quebec as a ``distinct society''. The politics of recognition, then, formed a way of making sense of settler-colonial relationships both with Indigenous peoples and within the multiculturalism of the settler-colonial powers themselves. With respect to both Indigenous and Quebecois sovereignty, part of the issue rested on the question whether Canada was a ``multicultural'' society - thereby reducing Indigenous and Quebecois social formations as solely ``cultural'' - or whether it was ``multinational'', and perhaps most importantly whether Canada could accommodate not only cultural and national diversity, but diversity in the mode of production (a challenge posed by Indigenous sovereignty and land claims). The politics of recognition was in one sense proposed to uphold minority collective or group rights, but in other away they were proposed to restrict diversity to the cultural sphere (hence the focus on ``multiculturalism'' over ``multinationalism'' and the explicit limiting of diversity to ``cultural diversity'' and ``cultural recognition'' in Kymlicka and Tully). In other words, as Coulthard points out, the politics of recognition was intended to save multiculturalism while removing the threat of multinationalism (or alternative modes of production), that is - to use Nancy Fraser's terms - to engage with affirmative vs. transformational modes of change (i.e. recognition, but not redistribution).

In 1985, the Canadian Government passed the Multiculturalism Act, which enshrined the politics of recognition as it applied to ``multicultural heritage'' and the ``rights of aboriginal peoples of Canada''. 

The Politics of Recognition comes up against a range of problems: the problem of unequal power relations identified by Coulthard, the problem of mode of production (i.e. the limiting of recognition to ``culture'') and the when the differences to be recognized are not ``cultural'' (i.e. race, sexuality, gender, disability, etc). There's also the problem that ``recognition'' itself is idealist - it presumes that the mental operation of recognition will be enough to change the material structures of society. So I think there are two main critiques: the individualist critique and the idealist critique. 

In the early to mid 1990s, concerns around what Charles Taylor called the ``narcissism'' of individualistic neoliberalism provoked a constellation of responses. Firstly, there was the  development of a communitarian liberalism which sought to critique the individualism of classical liberalism and to reinstate a notion of collective responsibility, for example in Taylor's work on the sources of modern individualism and the role it played in modern society (\cite{taylor-1989, Taylor-1991}). Secondly, there was the work done by Will Kymlicka, Charles Taylor, and others to challenge the dominant ways of understanding multiculturalism and minority rights (\cite{kymlicka-citizenship, taylor-1992, honneth-1995, fraser-1997}. Thirdly, there was the work done by James Tully on constitutionalism and diversity (\cite{tully-1995}). All of this work not only engaged with the changed nature of multicultural and multinational nation states with neoliberal turn, but also with the rise of identity-based social movements in the 1960s (and most especially after 1968), and with the end of the threat posed by the Eastern bloc countries which had been seen as anti-individualistic. The collapse of the Soviet Union opened up space for a collectivist critique of the individualism unleashed by neoliberal politics and economics, especially in the acquisitive decade par excellence, the 1980s. Furthermore, in the Canadian context, renewed Indigenous and Quebecois resistance to Canadian assimilation - the Kanehsatà:ke uprising in 1990 and the referendum on Quebecois independence in 1995 - for example, brought issues of assimilation vs. respect for minority rights very much to the forefront of Canadian cultural and political debates. The work of Kymlicka, Taylor, Tully, and others was an attempt to deal with the new realities of a unipolar, post-colonial, multiethnic and multinational polity. What they had in common was an attempt to update liberal theory to account for the post-welfare state realities of neoliberalism.

Patten (\cite{patten-2014} identified two broad tendencies in liberal theory with respect to minority rights. On the one hand there is the ``liberal culturalist'' position adhered to by both Kymlicka and Taylor, which takes the need to protect collective minority rights seriously, while on the other hand is an older liberalism which derives mainly from the work of Rawls, and which sees collective rights as fully reducible to individual rights, and which Patten identifies with the work of Waldron, Barry, Appiah, and others \cite[4]{patten-2014}. Besides maintaining that the older liberalism is fully adequate to the protection of minority rights, this position points to potential drawbacks of the liberal culturalist programme, such as the impossibility of national solidarity and the fragmentation of social cohesion into special interest groups. In librarianship, the older, Rawlsian liberalism tends to hold sway, while in Canada - due precisely to debates around Canadian multiculturalism, Indigenous and Quebecois sovereignty - the liberal culturalist position is predominant.

Within the liberal culturalist position, we can identify two tendencies, one towards a ``politics of difference'', for example in the work of Kymlicka, and one towards a ``politics of recognition'' formalized by Taylor and adopted by Tully. For Kymlicka, the permanent differentiation of collective minority rights anathema to more traditional liberals was based on a mistaken understanding of the term ``collective rights''. Kymlicka distinguishes between the rights of a group that ``limit[s] the liberty of its own individual members in the name of group solidarity or cultural purity'' and ``the right of a group to limit the economic or political power exercised by the larger society over the group'', and concludes that this second form of collective right need not conflict with the traditional liberal rights of individuals.


\backmatter

\bibliographystyle{plainnat} 
\bibliography{Bibliography}
\end{document}